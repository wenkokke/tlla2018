\documentclass{article}
\usepackage{amssymb}
\usepackage{hyperref}
\usepackage{graphicx}
\usepackage{xspace}
\usepackage{bm}
\usepackage{mathtools}
\usepackage{multirow}
\usepackage{cmll}
\def\one{\mathrel{%
    \mathchoice{\ONE}{\ONE}{\scriptsize\ONE}{\tiny\ONE}%
}}
\def\ONE{{%
    \setbox0\hbox{\hphantom{\bot}}%
    \rlap{\hbox to \wd0{\hss1\hss}}\box0
}}
\usepackage{bussproofs}
\EnableBpAbbreviations

\newcommand{\email}[1]{\href{mailto:#1}{#1}}
 
\DeclareGraphicsRule{.ai}{pdf}{.ai}{}
\newcommand{\asset}[1]{\ensuremath{\vcenter{\hbox{%
\includegraphics[width=1em]{assets/#1}}}}\xspace}

\newcommand{\mail}{\asset{2709.ai}}
\newcommand{\mailtok}{\asset{1f4e9.ai}}
\newcommand{\working}{\asset{1f469-1f3fb-200d-1f4bb.ai}}
\newcommand{\dunno}{\asset{2753.ai}}
\newcommand{\climbing}{\asset{1f9d7-1f3fb-200d-2640-fe0f.ai}}
\newcommand{\whatever}{\asset{1f937-1f3fb-200d-2640-fe0f.ai}}
\newcommand{\supervisor}{\asset{1f9d9-1f3fd-200d-2642-fe0f.ai}}

\newcommand{\ga}[1]{\bm{\underline{\color{mDarkTeal}\ensuremath{#1}}}}
\newcommand{\gb}[1]{\bm{\underline{\color{mDarkBrown}\ensuremath{#1}}}}

\providecommand{\ppar}{\ensuremath{\mid}}

\providecommand{\piSend}[3]{\ensuremath{#1\langle #2 \rangle.#3}}
\providecommand{\piBoundSend}[3]{\ensuremath{#1[ #2 ].#3}}
\providecommand{\piRecv}[3]{\ensuremath{#1( #2 ).#3}}
\providecommand{\piPar}[2]{\ensuremath{#1 \ppar #2}}
\providecommand{\piNew}[2]{\ensuremath{(\nu #1)#2}}
\providecommand{\piRepl}[1]{\ensuremath{!#1}}
\providecommand{\piHalt}[0]{\ensuremath{0}}
\providecommand{\piSub}[3]{\ensuremath{#3\{#1/#2\}}}

\providecommand{\tm}[1]{\ensuremath{\normalfont#1}}
\providecommand{\ty}[1]{\ensuremath{\normalfont#1}}
\providecommand{\seq}[2][]{\ensuremath{\tm{#1}\;\vdash\;\ty{#2}}}
\providecommand{\tmty}[2]{\ensuremath{\tm{#1}\colon\!\ty{#2}}}
\providecommand{\NOM}[1]{\RightLabel{\textsc{#1}}}
\providecommand{\SYM}[1]{\RightLabel{\ensuremath{#1}}}
\newenvironment{prooftree*}{\leavevmode\hbox\bgroup}{\DisplayProof\egroup}

\providecommand{\cpLink}[2]{\ensuremath{#1{\leftrightarrow}#2}}
\providecommand{\cpCut}[3]{\ensuremath{\piNew{#1}({\piPar{#2}{#3}})}}
\providecommand{\cpSend}[4]{\ensuremath{#1[#2].(\piPar{#3}{#4})}}
\providecommand{\cpRecv}[3]{\ensuremath{#1(#2).#3}}
\providecommand{\cpWait}[2]{\ensuremath{#1().#2}}
\providecommand{\cpHalt}[1]{\ensuremath{#1[].0}}
\providecommand{\cpInl}[2]{\ensuremath{#1\triangleleft\texttt{inl}.#2}}
\providecommand{\cpInr}[2]{\ensuremath{#1\triangleleft\texttt{inr}.#2}}
\providecommand{\cpCase}[3]{\ensuremath{#1\triangleright\{\texttt{inl}:#2;\texttt{inr}:#3\}}}
\providecommand{\cpAbsurd}[1]{\ensuremath{#1\triangleright\{\}}}
\providecommand{\cpSub}[3]{\ensuremath{\piSub{#1}{#2}{#3}}}
\providecommand{\cpPlug}[2]{\ensuremath{{#1}[{#2}]}}

\providecommand{\plus}{\ensuremath{\oplus}}
\providecommand{\tens}{\ensuremath{\otimes}}
\providecommand{\one}{\ensuremath{\mathbf{1}}}
\providecommand{\nil}{\ensuremath{\mathbf{0}}}
\providecommand{\limp}{\ensuremath{\multimap}}
\providecommand{\emptycontext}{\ensuremath{\,\cdot\,}}
\providecommand{\bigtens}{\ensuremath{\scalerel*{\tens}{\sum}}}
\providecommand{\bigparr}{\ensuremath{\scalerel*{\parr}{\sum}}}

\providecommand{\cpInfAx}{%
  \begin{prooftree*}
    \AXC{$\vphantom{\seq[ Q ]{ \Delta, \tmty{y}{A^\bot} }}$}
    \NOM{Ax}
    \UIC{$\seq[ \cpLink{x}{y} ]{ \tmty{x}{A}, \tmty{y}{A^\bot} }$}
  \end{prooftree*}}
\providecommand{\cpInfCut}{%
  \begin{prooftree*}
    \AXC{$\seq[ P ]{ \Gamma, \tmty{x}{A} }$}
    \AXC{$\seq[ Q ]{ \Delta, \tmty{x}{A^\bot} }$}
    \NOM{Cut}
    \BIC{$\seq[ \cpCut{x}{P}{Q} ]{ \Gamma, \Delta }$}
  \end{prooftree*}}
\providecommand{\cpInfMix}{%
  \begin{prooftree*}
    \AXC{$\seq[ P ]{ \Gamma }$}
    \AXC{$\seq[ Q ]{ \Delta }$}
    \NOM{Mix}
    \BIC{$\seq[ \piPar{P}{Q} ]{ \Gamma , \Delta }$}
  \end{prooftree*}}
\providecommand{\cpInfCycle}{%
  \begin{prooftree*}
    \AXC{$\seq[ P ]{ \Gamma, \tmty{x}{A}, \tmty{y}{A^\bot}}$}
  \end{prooftree*}}
\providecommand{\cpInfHalt}{%
  \begin{prooftree*}
    \AXC{$\vphantom{\seq[ Q ]{ \Delta, \tmty{y}{A^\bot} }}$}
    \NOM{Halt}
    \UIC{$\seq[{ \piHalt }]{}$}
  \end{prooftree*}}

%% Logical rules
\providecommand{\cpInfTens}{%
  \begin{prooftree*}
    \AXC{$\seq[ P ]{ \Gamma , \tmty{y}{A} }$}
    \AXC{$\seq[ Q ]{ \Delta , \tmty{x}{B} }$}
    \SYM{(\tens)}
    \BIC{$\seq[ \cpSend{x}{y}{P}{Q} ]{ \Gamma , \Delta , \tmty{x}{A \tens B} }$}
  \end{prooftree*}}
\providecommand{\cpInfParr}{%
  \begin{prooftree*}
    \AXC{$\seq[ P ]{ \Gamma , \tmty{y}{A} , \tmty{x}{B} }$}
    \SYM{(\parr)}
    \UIC{$\seq[ \cpRecv{x}{y}{P} ]{ \Gamma , \tmty{x}{A \parr B} }$}
  \end{prooftree*}}
\providecommand{\cpInfOne}{%
  \begin{prooftree*}
    \AXC{$\vphantom{\seq[ P ]{ \Gamma }}$}
    \SYM{(\one)}
    \UIC{$\seq[ \cpHalt{x} ]{ \tmty{x}{\one} }$}
  \end{prooftree*}}
\providecommand{\cpInfBot}{%
  \begin{prooftree*}
    \AXC{$\seq[ P ]{ \Gamma }$}
    \SYM{(\bot)}
    \UIC{$\seq[ \cpWait{x}{P} ]{ \Gamma , \tmty{x}{\bot} }$}
  \end{prooftree*}}
\providecommand{\cpInfPlus}[1]{%
  \ifdim#1pt=1pt
  \begin{prooftree*}
    \AXC{$\seq[ P ]{ \Gamma , \tmty{x}{A} }$}
    \SYM{(\plus_1)}
    \UIC{$\seq[{ \cpInl{x}{P} }]{ \Gamma , \tmty{x}{A \plus B} }$}
  \end{prooftree*}
  \else%
  \ifdim#1pt=2pt
  \begin{prooftree*}
    \AXC{$\seq[ P ]{ \Gamma , \tmty{x}{B} }$}
    \SYM{(\plus_2)}
    \UIC{$\seq[ \cpInr{x}{P} ]{ \Gamma , \tmty{x}{A \plus B} }$}
  \end{prooftree*}
  \else%
  \fi%
  \fi%
}
\providecommand{\cpInfWith}{%
  \begin{prooftree*}
    \AXC{$\seq[ P ]{ \Gamma , \tmty{x}{A} }$}
    \AXC{$\seq[ Q ]{ \Gamma , \tmty{x}{B} }$}
    \SYM{(\with)}
    \BIC{$\seq[ \cpCase{x}{P}{Q} ]{ \Gamma , \tmty{x}{A \with B} }$}
  \end{prooftree*}}
\providecommand{\cpInfNil}{%
  \text{(no rule for \ty{\nil})}}
\providecommand{\cpInfTop}{%
  \begin{prooftree*}
    \AXC{}
    \SYM{(\top)}
    \UIC{$\seq[ \cpAbsurd{x} ]{ \Gamma, \tmty{x}{\top} }$}
  \end{prooftree*}}

%%% Local Variables:
%%% TeX-master: "main"
%%% End:
\input{preamble-biblatex}
\input{preamble-theorems}
\input{preamble-typing}
\input{preamble-emoji}
\pagenumbering{gobble}
\usepackage[section]{placeins}
\usepackage[parfill]{parskip}
\usepackage{titlesec}
\newcommand{\sectionbreak}{\ifnum\value{section}>1\clearpage\fi}
\addbibresource{main.bib}
\author{Wen Kokke}
\title{Hypersequent Classical Processes}
\begin{document}


\section{Classical Processes}\label{sec:cp}
In this section, we will introduce a variant of \cp which removes the
commutative conversions from the reductions.
\cp was introduced by \textcite{wadler2012}, based on work by \textcite{caires2010}
and \textcite{gay2009}.

\subsection{Terms}\label{sec:cp-terms}
\input{def-cp-terms}
\begin{definition}[Structural congruence]\label{def:cp-equiv}
  We define the structural congruence $\equiv$ as the congruence closure over
  terms which satisfies the following additional axioms:
  \begin{gather*}
    \begin{array}{llcll}
      \cpEquivLinkComm
      & \tm{\cpLink{x}{y}}
      & \equiv
      & \tm{\cpLink{y}{x}}
      \\
      \cpEquivCutComm
      & \tm{\cpCut{x}{P}{Q}}
      & \equiv
      & \tm{\cpCut{x}{Q}{P}}
      \\
      \cpEquivCutAss1
      & \tm{\cpCut{x}{P}{\cpCut{y}{Q}{R}}}
      & \equiv
      & \tm{\cpCut{y}{\cpCut{x}{P}{Q}}{R}}
      & \text{if }\notFreeIn{x}{R}\text{ and }\notFreeIn{y}{P}
    \end{array}
  \end{gather*}
\end{definition}
%%% Local Variables:
%%% TeX-master: "main"
%%% End:

\begin{definition}[Reduction]\label{def:cp-reduction}
  A reduction $\reducesto{P}{Q}$ denotes that the process $\tm{P}$ can reduce
  to the process $\tm{Q}$ in a single step. Reductions can only be constructed
  as follows:
  \begin{gather*}
    \begin{array}{llcll}
      \cpRedAxCut1
      & \tm{\cpCut{x}{\cpLink{w}{x}}{P}}
      & \Longrightarrow
      & \tm{\cpSub{w}{x}{P}} 
      \\
      \cpRedBetaTensParr
      & \tm{\cpCut{x}{\cpSend{x}{y}{P}{Q}}{\cpRecv{x}{z}{R}}}
      & \Longrightarrow
      & \tm{\cpCut{y}{P}{\cpCut{x}{Q}{\cpSub{y}{z}{R}}}}
      \\
      \cpRedBetaOneBot
      & \tm{\cpCut{x}{\cpHalt{x}}{\cpWait{x}{P}}}
      & \Longrightarrow
      & \tm{P}
      \\
      \cpRedBetaPlusWith1
      & \tm{\cpCut{x}{\cpInl{x}{P}}{\cpCase{x}{Q}{R}}}
      & \Longrightarrow
      & \tm{\cpCut{x}{P}{Q}}
      \\
      \cpRedBetaPlusWith2
      & \tm{\cpCut{x}{\cpInr{x}{P}}{\cpCase{x}{Q}{R}}}
      & \Longrightarrow
      & \tm{\cpCut{x}{P}{R}}
    \end{array}
  \end{gather*}
  \begin{center}
    \begin{prooftree*}
      \AXC{$\reducesto{P}{P^\prime}$}
      \SYM{\cpRedGammaCut}
      \UIC{$\reducesto{\cpCut{x}{P}{Q}}{\cpCut{x}{P^\prime}{Q}}$}
    \end{prooftree*}
    \begin{prooftree*}
      \AXC{$\tm{P}\equiv\tm{Q}$}
      \AXC{$\reducesto{Q}{Q^\prime}$}
      \AXC{$\tm{Q^\prime}\equiv\tm{P^\prime}$}
      \SYM{\cpRedGammaEquiv}
      \TIC{$\reducesto{P}{P^\prime}$}
    \end{prooftree*}
  \end{center}
  The relations $\Longrightarrow^{?}$, $\Longrightarrow^{+}$, and
  $\Longrightarrow^\star$ are the reflexive, the transitive, and the reflexive,
  transitive closures of $\Longrightarrow$.
\end{definition}
%%% Local Variables:
%%% TeX-master: "main"
%%% End:


\subsection{Types}\label{sec:cp-types}
\input{def-cp-types}
\input{def-cp-negation}
\input{lem-cp-negation-involutive}
\begin{definition}[Environments]\label{def:cp-envs}
  We define environments as follows:
  \begin{gather*}
    \ty{\Gamma}, \ty{\Delta}, \ty{\Theta} ::=
    \tmty{x_1}{A_1}\dots\tmty{x_n}{A_n}
  \end{gather*}
  The empty environment is denoted $\emptycontext$.

  Names in environments must be unique, and environments $\ty{\Gamma}$ and
  $\ty{\Delta}$ can only be combined as $\ty{\Gamma}, \ty{\Delta}$ if
  $\text{fv}(\ty{\Gamma}) \cap \text{fv}(\ty{\Delta}) = \varnothing$.
\end{definition}
%%% Local Variables:
%%% TeX-master: "main"
%%% End:

\begin{definition}[Typing judgements]\label{def:cp-typing}
  A typing judgement $\seq[{ P }]{\tmty{x_1}{A_1}\dots\tmty{x_n}{A_n}}$ denotes
  that the process $\tm{P}$ communicates along channels $\tm{x_1}\dots\tm{x_n}$
  following protocols $\ty{A_1}\dots\ty{A_n}$. 
  Typing judgements can be constructed using the inference rules in
  \cref{fig:cp}.
\end{definition}
\begin{figure*}[!htb]
  Structural Rules.
  \begin{center} \cpInfAx     \cpInfCut      \end{center}\vspace*{1\baselineskip}

  Logical Rules.
  \begin{center} \cpInfTens   \cpInfParr     \end{center}\vspace*{1\baselineskip}
  \begin{center} \cpInfOne    \cpInfBot      \end{center}\vspace*{1\baselineskip}
  \begin{center} \cpInfPlus1  \cpInfPlus2    \end{center}\vspace*{1\baselineskip}
  \begin{center} \cpInfWith                  \end{center}\vspace*{1\baselineskip}
  \begin{center} \cpInfNil    \cpInfTop      \end{center}

  \caption{Classical Processes}
  \label{fig:cp}
\end{figure*}
%%% Local Variables:
%%% TeX-master: "main"
%%% End:


\subsection{Properties}\label{sec:cp-properties}

\subsubsection{Preservation}\label{sec:cp-preservation}
\begin{lemma}\label{lem:cp-preservation-equiv}
  If $\tm{P}\equiv\tm{Q}$, then $\seq[P]{\Gamma}$ iff $\seq[Q]{\Gamma}$.
\end{lemma}
\begin{proof}
  By induction on the structure of $\tm{P}\equiv\tm{Q}$.
  \begin{itemize}
  \item
    Case \cpEquivLinkComm.
    We have to show $\seq[\cpLink{x}{y}]{\Gamma} \Leftrightarrow
    \seq[\cpLink{y}{x}]{\Gamma}$.
    For each direction, we continue by inversion on the typing derivation of the premise.
    The only typing derivation ends in \textsc{Ax}.
    For each direction, we rewrite as follows:
    \[
      \begin{array}{lcl}
        \AXC{}
        \NOM{Ax}
        \UIC{$\seq[\cpLink{x}{y}]{\tmty{x}{A}, \tmty{y}{A^\bot}}$}
        \DisplayProof
        & \Longrightarrow
        & \AXC{}
          \NOM{Ax}
          \UIC{$\seq[\cpLink{y}{x}]{\tmty{y}{A^\bot}, \tmty{x}{A^{\bot\bot}}}$}
          \NOM{\cref{lem:cp-negation-involutive}}
          \UIC{$\seq[\cpLink{y}{x}]{\tmty{y}{A^\bot}, \tmty{x}{A}}$}
          \DisplayProof
      \end{array}
    \]
  \item
    Case \cpEquivCutComm.
    We have to show $\seq[\cpCut{x}{P}{Q}]{\Gamma} \Leftrightarrow
    \seq[\cpCut{x}{Q}{P}]{\Gamma}$.
    For each direction, we continue by inversion on the typing derivation of the
    premise.
    The only typing derivation ends in \textsc{Cut}.
    We rewrite as follows:
    \[
      \begin{array}{lcl}
        \AXC{$\seq[P]{\Gamma}$}
        \AXC{$\seq[Q]{\Delta}$}
        \NOM{Cut}
        \BIC{$\seq[\cpCut{x}{P}{Q}]{\Gamma \hsep \Delta}$}
        \DisplayProof
        & \Leftrightarrow
        & \AXC{$\seq[Q]{\Delta}$}
          \AXC{$\seq[P]{\Gamma}$}
          \NOM{Cut}
          \BIC{$\seq[\cpCut{x}{Q}{P}]{\Delta \hsep \Gamma}$}
          \DisplayProof
      \end{array}
    \]
  \item
    Case \cpEquivCutAss1.
    We have to show $\seq[\cpCut{x}{P}{\cpCut{x}{Q}{R}}]{\Gamma}
    \Leftrightarrow \seq[\cpCut{x}{\cpCut{x}{P}{Q}}{R}]{\Gamma}$.
    For each direction, we continue by inversion on the typing derivation of the
    premise. 
    The only typing derivation ends in two applications of \textsc{Cut}.
    We rewrite as follows:
    \[
      \begin{array}{c}
        \AXC{$\seq[P]{\Gamma, \tmty{x}{A}}$}
        \AXC{$\seq[Q]{\Delta, \tmty{x}{A^\bot}, \tmty{y}{B}}$}
        \AXC{$\seq[R]{\Theta, \tmty{y}{B^\bot}}$}
        \NOM{Cut}
        \BIC{$\seq[\cpCut{y}{Q}{R}]{ \Delta, \Theta, \tmty{x}{A^\bot}}$}
        \NOM{Cut}
        \BIC{$\seq[\cpCut{x}{P}{\cpCut{y}{Q}{R}}]{ \Gamma, \Delta, \Theta}$}
        \DisplayProof
        \\\\
        \Updownarrow
        \\\\
        \AXC{$\seq[P]{\Gamma, \tmty{x}{A}}$}
        \AXC{$\seq[Q]{\Delta, \tmty{x}{A^\bot}, \tmty{y}{B}}$}
        \NOM{Cut}
        \BIC{$\seq[\cpCut{x}{P}{Q}]{\Gamma, \Delta, \tmty{y}{B}}$}
        \AXC{$\seq[R]{\Theta, \tmty{y}{B^\bot}}$}
        \NOM{Cut}
        \BIC{$\seq[\cpCut{y}{\cpCut{x}{P}{Q}}{R}]{\Gamma, \Delta, \Theta}$}
        \DisplayProof
      \end{array}
    \]

  \end{itemize}
  The cases for reflexivity, transitivity, symmetry, and congruence are trivial.
\end{proof}
%%% Local Variables:
%%% TeX-master: "main"
%%% End:

\begin{lemma}\label{lem:cp-preservation-subst}
  If $\seq[P]{\tmty{x}{A}, \Gamma}$, then $\seq[\cpSub{y}{x}{P}]{\tmty{y}{A},
    \Gamma}$.
\end{lemma}
\begin{proof}
  By induction on the structure of the premise.
\end{proof}
%%% Local Variables:
%%% TeX-master: "main"
%%% End:

\begin{theorem}[Preservation]\label{thm:cp-preservation}
  If $\reducesto{P}{Q}$ and $\seq[P]{\Gamma}$, then $\seq[Q]{\Gamma}$.
\end{theorem}
\begin{proof}
  By induction on the structure of $\reducesto{P}{Q}$.
  \begin{itemize}
  \item
    Case \cpRedAxCut1.
    We have to show $\seq[\cpCut{x}{\cpLink{w}{x}}{R}]{\Gamma} \Rightarrow
    \seq[\cpSub{w}{x}{P}]{\Gamma}$. 
    We continue by inversion on the typing derivation of the premise.
    The only typing derivation ends in \textsc{Cut} and \textsc{Ax}.
    We rewrite as follows:
    \[
      \begin{array}{lcl}
        \AXC{}
        \NOM{Ax}
        \UIC{$\seq[\cpLink{w}{x}]{ \tmty{w}{A}, \tmty{x}{A^\bot} }$}
        \AXC{$\seq[P]{ \tmty{x}{A}, \Gamma }$}
        \NOM{Cut}
        \BIC{$\seq[\cpCut{x}{\cpLink{w}{x}}{P}]{ \tmty{w}{A}, \Gamma }$}
        \DisplayProof
        & \Rightarrow
        & \AXC{$\seq[P]{ \tmty{x}{A}, \Gamma }$}
          \NOM{\Cref{lem:cp-preservation-subst}}
          \UIC{$\seq[\cpSub{w}{x}{P}]{ \tmty{w}{A}, \Gamma }$}
          \DisplayProof
      \end{array}
    \]
  \item
    Case \cpRedBetaTensParr.
    We have to show $\seq[\cpCut{y}{\cpRecv{y}{x}{P}}{\cpSend{y}{x}{Q}{R}}]{
      \Gamma} \Rightarrow \seq[\cpCut{y}{\cpCut{x}{P}{Q}}{R}]{\Gamma}$. 
    We continue by inversion on the typing derivation of the premise.
    The only typing derivation ends in \textsc{Cut}, $(\parr)$, and $(\tens)$.
    We rewrite as follows:
    \[
      \begin{array}{c}
        \AXC{$\seq[P]{ \Gamma, \tmty{x}{A^\bot}, \tmty{y}{B^\bot} }$}
        \SYM{\parr}
        \UIC{$\seq[\cpRecv{y}{x}{P}]{ \Gamma, \tmty{y}{A^\bot \parr B^\bot} }$}
        \AXC{$\seq[Q]{ \Delta, \tmty{x}{A} }$}
        \AXC{$\seq[R]{ \Theta, \tmty{y}{B} }$}
        \SYM{\tens}
        \BIC{$\seq[\cpSend{y}{x}(Q \mid R)]{ \Delta, \Theta, \tmty{y}{A \tens B} }$}
        \NOM{Cut}
        \BIC{$\seq[\cpCut{y}{\cpRecv{y}{x}{P}}{\cpSend{y}{x}{Q}{R}}]{ \Gamma, \Delta, \Theta }$}
        \DisplayProof
        \\\\
        \Downarrow
        \\\\
        \AXC{$\seq[P]{ \Gamma, \tmty{x}{A^\bot}, \tmty{y}{B^\bot} }$}
        \AXC{$\seq[Q]{ \Delta, \tmty{x}{A} }$}
        \NOM{Cut}
        \BIC{$\seq[\cpCut{x}{P}{Q}]{ \Gamma, \Delta, \tmty{y}{B^\bot} }$}
        \AXC{$\seq[R]{ \Theta, \tmty{y}{B} }$}      
        \NOM{Cut}
        \BIC{$\seq[\cpCut{y}{\cpCut{x}{P}{Q}}{R}]{ \Gamma, \Delta, \Theta }$}
        \DisplayProof
      \end{array}
    \]
  \item
    Case \cpRedBetaOneBot.
    We have to show $\seq[\cpCut{x}{\cpHalt{x}}{\cpWait{x}{P}}]{\Gamma}
    \Rightarrow \seq[P]{\Gamma}$. 
    We continue by inversion on the typing derivation of the premise.
    The only typing derivation ends in \textsc{Cut}, $(\one)$, and $(\bot)$.
    We rewrite as follows:
    \[
      \begin{array}{lcl}
        \AXC{$\seq[P]{ \Gamma }$}
        \SYM{\bot}
        \UIC{$\seq[\cpWait{x}{P}]{ \Gamma, \tmty{x}{\bot} }$}
        \AXC{}
        \SYM{\one}
        \UIC{$\seq[\cpHalt{x}]{ \tmty{x}{\one} }$}
        \NOM{Cut}
        \BIC{$\seq[\cpCut{x}{\cpHalt{x}}{\cpWait{x}{P}}]{ \Gamma }$}
        \DisplayProof
        & \Rightarrow
        & \AXC{$\seq[P]{ \Gamma }$}
          \DisplayProof
      \end{array}
    \]
  \item
    Case \cpRedBetaPlusWith1.
    We have to show $\seq[\cpCut{x}{\cpCase{x}{P}{Q}}{\cpInl{x}{R}}]{\Gamma}
    \Rightarrow \seq[\cpCut{x}{P}{R}]{\Gamma}$.
    We continue by inversion on the typing derivation of the premise.
    The only typing derivation ends in \textsc{Cut}, $(\plus_1)$, and $(\with)$.
    We rewrite as follows:
    \[
      \begin{array}{c}
        \AXC{$\seq[P]{ \Gamma, \tmty{x}{A^\bot} }$}
        \AXC{$\seq[Q]{ \Gamma, \tmty{x}{B^\bot} }$}
        \SYM{\with}
        \BIC{$\seq[\cpCase{x}{P}{Q}]{ \Gamma, \tmty{x}{A^\bot \with B^\bot} }$}
        \AXC{$\seq[R]{ \Delta, \tmty{x}{A} }$}
        \SYM{\plus_1}
        \UIC{$\seq[\cpInl{x}{R}]{ \Delta, \tmty{x}{A \plus B} }$}
        \NOM{Cut}
        \BIC{$\seq[\cpCut{x}{\cpCase{x}{P}{Q}}{\cpInl{x}{R}}]{ \Gamma, \Delta }$}
        \DisplayProof
        \\\\
        \Downarrow
        \\\\
        \AXC{$\seq[P]{ \Gamma, \tmty{x}{A^\bot} }$}
        \AXC{$\seq[R]{ \Delta, \tmty{x}{A} }$}
        \NOM{Cut}
        \BIC{$\seq[\cpCut{x}{P}{R}]{ \Gamma, \Delta }$} 
        \DisplayProof
      \end{array}
    \]
  \item
    Case \cpRedBetaPlusWith2.
    As above.
  \end{itemize}
  The case for \cpRedGammaEquiv follows from \cref{lem:cp-preservation-equiv}.
  The cases for reflexivity, transitivity, \cpRedGammaCut are trivial.
\end{proof}
%%% Local Variables:
%%% TeX-master: "main"
%%% End:


\subsubsection{Canonical Forms and Progress}\label{sec:cp-canonical-forms-and-progress}

\paragraph{Canonical Forms}\label{sec:cp-canonical-forms}
\input{def-cp-action}

\paragraph{Evaluation Contexts}\label{sec:cp-evaluation-contexts}
\input{def-cp-evaluation-contexts}
\input{def-cp-evaluation-prefixes}
\input{def-cp-maximum-evaluation-prefixes}
\begin{lemma}\label{thm:cp-maximum-evaluation-prefix}
  Every term $\tm{P}$ has a maximum evaluation prefix.
\end{lemma}
\begin{proof}
  By induction on the structure of $\tm{P}$.
\end{proof}
%%% Local Variables:
%%% TeX-master: "main"
%%% End:


\paragraph{Progress}\label{sec:cp-progress}
\input{thm-cp-progress}


\section{Hypersequent Classical Processes}\label{sec:hcp}
In this section, we will introduce a variant of \cp based on hypersequent
calculi.
Hypersequent calculi were introduced independently by \textcite{avron1987} and
\textcite{pottinger1983}.

\subsection{Terms}\label{sec:hcp-terms}
\begin{definition}[Terms]\label{def:hcp-terms}
  We extend the terms from \cref{def:cp-terms} with the following construct:
  \begin{gather*}
    \begin{aligned}
      \tm{P}, \tm{Q}, \tm{R}
          :=& \; \dots
      \\\mid& \; \tm{\piHalt}          &&\text{terminated process}
      \\\mid& \; \tm{( \piPar{P}{Q} )} &&\text{parallel composition, or ``mix''}
    \end{aligned}
  \end{gather*}
\end{definition}
%%% Local Variables:
%%% TeX-master: "main"
%%% End:

\begin{definition}[Structural congruence]\label{def:hcp-equiv}
  We define the structural congruence $\equiv$ as the congruence closure over
  terms which satisfies the axioms in \cref{def:cp-equiv} and the following
  additional axioms:
  \begin{gather*}
    \begin{array}{llcll}
      \hcpEquivMixComm
      & \tm{\piPar{P}{Q}}
      & \equiv
      & \tm{\piPar{Q}{P}}
      \\
      \hcpEquivMixAss1
      & \tm{\piPar{P}{( \piPar{Q}{R}} )}
      & \equiv
      & \tm{\piPar{( \piPar{P}{Q} )}{R}}
      \\
      \hcpEquivMixCut1
      & \tm{\cpCut{x}{( \piPar{P}{Q} )}{R}}
      & \equiv
      & \tm{\piPar{P}{\cpCut{x}{Q}{R}}}
      & \text{if }\notFreeIn{x}{P} 
      \\
      \hcpEquivMixHalt1
      & \tm{\piPar{P}{\piHalt}}
      & \equiv
      & \tm{P}
    \end{array}
  \end{gather*}
\end{definition}
%%% Local Variables:
%%% TeX-master: "main"
%%% End:

\begin{definition}[Reduction]\label{def:hcp-reduction}
  A reduction $\reducesto{P}{Q}$ denotes that the process $\tm{P}$ can reduce
  to the process $\tm{Q}$ in a single step. Reductions can only be constructed
  using the rules in \cref{def:cp-reduction} and the following addition
  congruence rule:
  \begin{prooftree}
    \AXC{$\reducesto{P}{P^\prime}$}
    \SYM{\hcpRedGammaMix}
    \UIC{$\reducesto{\piPar{P}{Q}}{\piPar{P^\prime}{Q}}$}
  \end{prooftree}
  The relations $\Longrightarrow^{?}$, $\Longrightarrow^{+}$, and
  $\Longrightarrow^\star$ are the reflexive, transitive and reflexive,
  transitive closures of $\Longrightarrow$.
\end{definition}
%%% Local Variables:
%%% TeX-master: "main"
%%% End:


\subsection{Types}\label{sec:hcp-types}
We keep the types and environments from \cp, see \cref{def:cp-types} and
\cref{def:cp-envs}.
We introduce a new layer on top of sequents: hypersequents.
However, as \cp is a one-sided logic, and it uses the left-hand side of the
turnstile to write the process, the traditional hypersequent notation can look
awkward:
\[
  \seq[P]{\Gamma_1}\hsep\dots\hsep\seq{\Gamma_n}.
\]
The above may be confusing. It seems to claim that $\tm{P}$ acts according to
protocol $\ty{\Gamma_1}$. What are all the other $\Gamma$s doing there?
Are they typing empty processes?
Therefore, we opt to leave out the repeated turnstile, and instead work with the
notion of ``hyper-environments''. However, we will still refer to our system as a
hypersequent system.
\begin{definition}[Hyper-environments]\label{def:hcp-hyperenvs}
  We define hyper-environments as follows:
  \begin{gather*}
    \ty{\mathcal{G}}, \ty{\mathcal{H}}
    ::= \ty{\Gamma_1 \hsep \dots \hsep \Gamma_n}
  \end{gather*}
  The empty hyper-environment is denoted $\ty{\emptyhypercontext}$.

  Note that variable names may occur multiple times in a hyper-environment, as
  long as names within each environment are still unique.
\end{definition}
%%% Local Variables:
%%% TeX-master: "main"
%%% End:

\begin{definition}[Typing judgements]\label{def:hcp-typing}
  A typing judgement $\seq[P]{\Gamma_1 \hsep \dots \hsep \Gamma_n}$ denotes
  that the process $\tm{P}$ consists of $n$ independent, but potentially
  interleaved processes, each of which communicates according to its own
  protocol $\Gamma_i$. 
  Typing judgements can be constructed using the inference rules in
  \cref{fig:hcp}.
\end{definition}
\begin{figure*}[!htb]
  Structural Rules.
  \begin{center} \hcpInfAx     \hcpInfCut    \end{center}\vspace*{1\baselineskip}
  \begin{center} \hcpInfMix    \hcpInfHalt   \end{center}\vspace*{1\baselineskip}
  
  Logical Rules.
  \begin{center} \hcpInfTens   \hcpInfParr   \end{center}\vspace*{1\baselineskip}
  \begin{center} \hcpInfOne    \hcpInfBot    \end{center}\vspace*{1\baselineskip}
  \begin{center} \hcpInfPlus1  \hcpInfPlus2  \end{center}\vspace*{1\baselineskip}
  \begin{center} \hcpInfWith                 \end{center}\vspace*{1\baselineskip}
  \begin{center} \hcpInfNil    \hcpInfTop    \end{center} 
  
  \caption{Hypersequent Classical Processes with \textsc{H-Mix}}
  \label{fig:hcp}
\end{figure*}
%%% Local Variables:
%%% TeX-master: "main"
%%% End:


\subsection{Properties}\label{sec:hcp-properties}

\subsubsection{Preservation}\label{sec:hcp-preservation}
\begin{lemma}\label{lem:hcp-preservation-equiv}
  If $\tm{P}\equiv\tm{Q}$, then $\seq[P]{\mathcal{G}}$ iff $\seq[Q]{\mathcal{G}}$.
\end{lemma}
\begin{proof}
  By induction on the structure of $\tm{P}\equiv\tm{Q}$.
  \begin{itemize}
  \item
    Case \hcpEquivLinkComm, \hcpEquivCutComm, and \hcpEquivCutAss1.
    As \cref{lem:cp-preservation-equiv}.
  \item
    Case \hcpEquivMixComm.
    We have to show $\seq[\piPar{P}{Q}]{\mathcal{G}} \Rightarrow
    \seq[\piPar{Q}{P}]{\mathcal{G}}$.
    For each direction, we continue by inversion on the typing derivation of the
    premise.
    The only typing derivation ends in \textsc{H-Mix}.
    We rewrite as follows:
    \[
      \begin{array}{lcl}
        \AXC{$\seq[P]{ \mathcal{G}_1 }$}
        \AXC{$\seq[Q]{ \mathcal{G}_2 }$}
        \NOM{H-Mix}
        \BIC{$\seq[\piPar{P}{Q}]{ \mathcal{G}_1 \hsep \mathcal{G}_2 }$}
        \DisplayProof
        & \Leftrightarrow
        & \AXC{$\seq[Q]{ \mathcal{G}_2 }$}
          \AXC{$\seq[P]{ \mathcal{G}_1 }$}
          \NOM{H-Mix}
          \BIC{$\seq[\piPar{Q}{P}]{ \mathcal{G}_1 \hsep \mathcal{G}_2 }$}
          \DisplayProof
      \end{array}
    \]
  \item
    Case \hcpEquivMixAss1.
    We have to show $\seq[\piPar{(\piPar{P}{Q})}{R}]{\mathcal{G}} \Rightarrow
    \seq[\piPar{(\piPar{P}{Q})}{R}]{\mathcal{G}}$.
    For each direction, we continue by inversion on the typing derivation of the
    premise.
    The only typing derivation ends in two applications of \textsc{H-Mix}.
    We rewrite as follows:
    \[
      \begin{array}{c}
        \AXC{$\seq[P]{ \mathcal{G}_1 }$}
        \AXC{$\seq[Q]{ \mathcal{G}_2 }$}
        \NOM{H-Mix}
        \BIC{$\seq[\piPar{P}{Q}]{ \mathcal{G}_1 \hsep \mathcal{G}_2 }$}
        \AXC{$\seq[R]{ \mathcal{G}_3 }$}
        \NOM{H-Mix}
        \BIC{$\seq[\piPar{(\piPar{P}{Q})}{R}]{ \mathcal{G}_1 \hsep
        \mathcal{G}_2 \hsep \mathcal{G}_3 }$}
        \DisplayProof
        \\\\
        \Updownarrow
        \\\\
        \AXC{$\seq[P]{ \mathcal{G}_1 }$}
        \AXC{$\seq[Q]{ \mathcal{G}_2 }$}
        \AXC{$\seq[R]{ \mathcal{G}_3 }$}
        \NOM{H-Mix}
        \BIC{$\seq[\piPar{Q}{R}]{ \mathcal{G}_2 \hsep \mathcal{G}_3 }$}
        \NOM{H-Mix}
        \BIC{$\seq[\piPar{(\piPar{P}{Q})}{R}]{ \mathcal{G}_1 \hsep
        \mathcal{G}_2 \hsep \mathcal{G}_3 }$}
        \DisplayProof
      \end{array}
    \]
  \item
    Case \hcpEquivMixCut1.
    We have to show $\seq[\piPar{P}{\cpCut{x}{Q}{R}}]{\mathcal{G}} \Rightarrow
    \seq[\piPar{P}{\cpCut{x}{Q}{R}}]{\mathcal{G}}$.
    For each direction, we continue by inversion on the typing derivation of the
    premise.
    The only typing derivation ends in \textsc{H-Mix} and \textsc{Cut}.
    We rewrite as follows:
    \[
      \begin{array}{c}
        \AXC{$\seq[P]{ \mathcal{G}_1 }$}
        \AXC{$\seq[Q]{ \mathcal{G}_2 \hsep \Gamma, \tmty{x}{A} }$}
        \AXC{$\seq[R]{ \mathcal{G}_3 \hsep \Delta, \tmty{x}{A^\bot} }$}
        \NOM{Cut}
        \BIC{$\seq[\cpCut{x}{Q}{R}]{ \mathcal{G}_2 \hsep \mathcal{G}_3 \hsep
        \Gamma, \Delta }$}
        \NOM{H-Mix}
        \BIC{$\seq[\piPar{P}{\cpCut{x}{Q}{R}}]{ \mathcal{G}_1 \hsep
        \mathcal{G}_2 \hsep \mathcal{G}_3 \hsep \Gamma, \Delta }$}
        \DisplayProof
        \\\\
        \Updownarrow
        \\\\
        \AXC{$\seq[P]{ \mathcal{G}_1 }$}
        \AXC{$\seq[Q]{ \mathcal{G}_2 \hsep \Gamma, \tmty{x}{A} }$}
        \NOM{H-Mix}
        \BIC{$\seq[\piPar{P}{\cpCut{x}{Q}{R}}]{ \mathcal{G}_1 \hsep
        \mathcal{G}_2 \hsep \Gamma, \tmty{x}{A} }$}
        \AXC{$\seq[R]{ \mathcal{G}_3 \hsep \Delta, \tmty{x}{A^\bot} }$}
        \NOM{Cut}
        \BIC{$\seq[\piPar{P}{\cpCut{x}{Q}{R}}]{ \mathcal{G}_1 \hsep
        \mathcal{G}_2 \hsep \mathcal{G}_3 \hsep \Gamma, \Delta }$}
        \DisplayProof
      \end{array}
    \]
  \item
    Case \hcpEquivMixHalt1.
    We have to show that if $\seq[(\piPar{P}{\piHalt})]{\mathcal{G}}$, then
    $\seq[P]{\mathcal{G}}$, and the converse.
    In the $[\Rightarrow]$ direction, we proceed by inversion on the typing
    derivation of the premise.
    The only derivation ends in \textsc{H-Mix} and \textsc{H-Halt}.
    In the $[\Leftarrow]$ direction, we rewrite immediately.
    For either direction, we rewrite as follows:
    \[
      \begin{array}{lcl}
        \AXC{$\seq[P]{\mathcal{G}}$}
        \AXC{}
        \NOM{H-Halt}
        \UIC{$\seq[\piHalt]{\emptyhypercontext}$}
        \NOM{H-Mix}
        \BIC{$\seq[(\piPar{P}{\piHalt})]{\mathcal{G}}$}
        \DisplayProof
        & \Leftrightarrow
        & \AXC{$\seq[P]{\mathcal{G}}$}
          \DisplayProof
      \end{array}
    \]
  \end{itemize}
\end{proof}
%%% Local Variables:
%%% TeX-master: "main"
%%% End:


\input{lem-hcp-preservation-subst}
\begin{theorem}[Preservation]\label{thm:hcp-preservation}
  If $\seq[P]{\mathcal{G}}$ and $\reducesto{P}{Q}$, then $\seq[Q]{\mathcal{G}}$.
\end{theorem}
\begin{proof}
  As \cref{thm:cp-preservation}.
  The additional case for \hcpRedGammaMix is trivial.
\end{proof}
%%% Local Variables:
%%% TeX-master: "main"
%%% End:


\subsubsection{Canonical Forms and Progress}\label{sec:hcp-canonical-forms-and-progress}

\paragraph{Canonical Forms}\label{sec:hcp-canonical-forms}
\begin{definition}[Canonical forms]\label{def:hcp-canonical-forms}
  A process $\tm{P}$ is in canonical form if it is an action, or if it is of the
  form
  \[
    \tm{(\piPar{
        \piNew{x_1}{(\piPar{P_1}{\dots \piNew{x_n}{(\piPar{P_n}{P_{n+1}})} \dots})}
      }{
        Q_1 \ppar \dots \ppar Q_m
      })},
  \]
  where each $\tm{P_i}$ and each $\tm{Q_i}$ is an action, no $\tm{P_i}$ is a
  link acting on a bound channel, and no two actions $\tm{P_i}$ and $\tm{P_j}$
  act on the same bound channel.

  An immediate consequence of this definition is that if a process is in
  canonical form, then at least one of the actions $\tm{P_i}$ and all of the
  actions $\tm{Q_i}$ are acting on free channels.
\end{definition}
%%% Local Variables:
%%% TeX-master: "main"
%%% End:


\paragraph{Evaluation contexts}
\begin{definition}[Evaluation contexts]\label{def:hcp-evaluation-contexts}
  We extend \cref{def:cp-evaluation-contexts} with the following constructs:
  \begin{align*}
    \tm{E} := \dots \mid \tm{( \piPar{E}{P} )} \mid \tm{( \piPar{P}{E} )}
  \end{align*}
\end{definition}
\begin{definition}[Plugging]\label{def:hcp-evaluation-context-plugging}
  We extend \cref{def:cp-evaluation-context-plugging} with the following cases:
  \begin{gather*}
    \begin{array}{lcl}
      \tm{\cpPlug{( \piPar{E}{P} )}{R}}
      & := & \tm{\piPar{( \cpPlug{E}{R} )}{P}}
      \\
      \tm{\cpPlug{( \piPar{P}{E} )}{R}}
      & := & \tm{\piPar{P}{( \cpPlug{E}{R} )}}
    \end{array}
  \end{gather*}
\end{definition}
%%% Local Variables:
%%% TeX-master: "main"
%%% End:

\begin{definition}[Evaluation prefixes]\label{def:hcp-evaluation-prefixes}
  We extend \cref{def:cp-evaluation-prefixes} with the following constructs:
  \begin{align*}
    \tm{G}, \tm{H} := \dots \mid \tm{(\piPar{G}{H})}
  \end{align*}
\end{definition}
\begin{definition}[Plugging]\label{def:hcp-evaluation-prefix-plugging}
  We define plugging for an evaluation prefix with $n$ holes as:
  \begin{gather*}
    \begin{array}{l}
      \tm{\cpPlug{(\piPar{G}{H})}{R_1 \dots R_m, R_{m+1} \dots R_{n}}}
      := \; \tm{(\piPar{\cpPlug{G}{R_1 \dots R_m}}{\cpPlug{H}{R_{m+1} \dots R_n}})}
    \end{array}
  \end{gather*}
  Note that in this case, $\tm{G}$ is an evaluation prefix with $m$ holes,
  and $\tm{H}$ is an evaluation prefix with $(n-m)$ holes.
\end{definition}
%%% Local Variables:
%%% TeX-master: "main"
%%% End:


\paragraph{Progress}
\begin{theorem}[Progress]\label{thm:hcp-progress}
  If $\seq[P]{\mathcal{G}}$, then there exists a $\tm{Q}$ such that either
  $\tm{P}\equiv\tm{Q}$ and $\tm{Q}$ is in canonical form, or $\reducesto{P}{Q}$.
\end{theorem}
\begin{proof}
  By induction on the structure of derivation for $\seq[{ P }]{ \mathcal{G} }$.
  The only interesting cases are when the last rule of the derivation is
  \textsc{Cut} or \textsc{H-Mix}. In every other case, the typing rule
  constructs a term which is in canonical form. 
  \\
  If the last rule in the derivation is \textsc{Cut} or \textsc{H-Mix}, we
  consider the maximum evaluation prefix $\tm{G}$ of $\tm{P}$, such that $\tm{P}
  = \tm{\cpPlug{G}{P_1 \dots P_{m+n+1}}}$ and each $\tm{P_i}$ is an action.
  The prefix $\tm{G}$ consists of $m$ mixes, $n$ cuts, and introduces $n$
  channels, but composes $m+n+1$ actions, at most $m+1$ of which are on the same
  side of all cuts.
  \begin{itemize}
  \item
    One of the actions composed by $\tm{G}$ is a link $\tm{\cpLink{x}{y}}$
    acting on a bound channel.
    Suppose that $\tm{x}$ is the bound channel.
    There exist evaluation contexts $\tm{E}$ and $\tm{E'}$ such that
    \[
      \tm{P} =
      \tm{\cpPlug{E}{\cpCut{x}{\cpPlug{E'}{\cpLink{x}{y}}}{R}}}.
    \]
    We rewrite by $\equiv$ to obtain
    \[
      \tm{P} =
      \tm{\cpPlug{E}{\cpPlug{E'}{\cpCut{x}{\cpLink{x}{y}}{R}}}}.
    \]
    We then reduce by applying \cpRedAxCut1.
    Similarly if $\tm{y}$ is the bound channel.
  \item
    Two of the actions composed by $\tm{G}$ are acting on the \emph{same} bound
    channel.
    Let the two actions be $\tm{P_i}$ and $\tm{P_j}$, and the bound channel
    $\tm{x}$.
    There exist evaluation contexts $\tm{E}$, $\tm{E_i}$, and $\tm{E_j}$ such
    that
    \[
      \tm{P} =
      \tm{\cpPlug{E}{\cpCut{x}{\cpPlug{E_i}{P_i}}{\cpPlug{E_j}{P_j}}}}
    \]
    We rewrite by $\equiv$ to obtain
    \[
      \tm{P} \equiv
      \tm{\cpPlug{E}{\cpPlug{E_i}{\cpPlug{E_j}{\cpCut{x}{P_i}{P_j}}}}}
    \]
    We then reduce by applying the appropriate \textbeta-reduction.
  \item
    Otherwise (at least) one of the processes acts on an external channel.
    No process $\tm{P_i}$ is a link acting on a bound channel.
    No two processes $\tm{P_i}$ and $\tm{P_j}$ act on the same channel $\tm{x}$.
    Therefore, $\tm{P}$ is equivalent to a process in canonical form.
  \end{itemize}
\end{proof}
%%% Local Variables:
%%% TeX-master: "main"
%%% End:


\subsection{Relation to \cp}\label{sec:hcp2cp}

\subsubsection{\hcp shares \textsc{Mix}-free processes with \cp}
First off, we have the following theorems, relating terms with a singleton
hypersequent, i.e.\ \textsc{Mix}-free terms, in \hcp to terms in \cp and vice
versa.
\begin{theorem}\label{thm:cp2hcp-typing}
  If $\seq[P]{\Gamma}$ in \cp, then $\seq[P]{\Gamma}$ in \hcp.
\end{theorem}
\begin{proof}
  By induction on the structure of the derivation of $\seq[P]{\Gamma}$.
\end{proof}

\begin{theorem}\label{thm:cp2hcp-equiv}
  If $\tm{P}\equiv\tm{Q}$ in \cp, then $\tm{P}\equiv\tm{Q}$ in \hcp.
\end{theorem}
\begin{proof}
  By induction on the derivation of $\tm{P}\equiv\tm{Q}$.
\end{proof}
\begin{theorem}\label{thm:cp2hcp-reduction}
  If $\reducesto{P}{Q}$ in \cp, then $\reducesto{P}{Q}$ in \hcp.
\end{theorem}
\begin{proof}
  By induction on the derivation of $\reducesto{P}{Q}$.
\end{proof}
\begin{theorem}\label{thm:hcp2cp-typing}
  If $\seq[P]{\Gamma}$ in \hcp, then $\seq[P]{\Gamma}$ in \cp.
\end{theorem}
\begin{proof}
  By induction on the structure of the derivation of $\seq[P]{\Gamma}$.
  From $\seq[P]{\Gamma}$ we conclude that $\tm{P}$ is \textsc{Mix}-free, and
  therefore all sequents in the derivation of $\seq[P]{\Gamma}$ have a single
  environment.
\end{proof}

\begin{theorem}\label{thm:hcp2cp-equiv}
  If $\seq[P]{\Gamma}$ and $\tm{P}\equiv\tm{Q}$ in \hcp, then $\tm{P}\equiv\tm{Q}$ in \cp.
\end{theorem}
\begin{proof}
  By induction on the structure of the derivation of $\tm{P}\equiv\tm{Q}$.
  From $\seq[P]{\Gamma}$ we conclude that $\tm{P}$ is \textsc{Mix}-free, and
  therefore none of the equivalences introduced in \hcp will occur in the
  derivation of $\tm{P}\equiv\tm{Q}$.
\end{proof}
\begin{theorem}\label{thm:hcp2cp-reduction}
  If $\seq[P]{\Gamma}$ and $\reducesto{P}{Q}$ in \hcp, then $\reducesto{P}{Q}$ in \cp.
\end{theorem}
\begin{proof}
  By induction on the structure of the derivation of $\reducesto{P}{Q}$.
  From $\seq[P]{\Gamma}$ we conclude that $\tm{P}$ is \textsc{Mix}-free, and
  therefore \hcpRedGammaMix will never occur in the derivation of $\reducesto{P}{Q}$.
\end{proof}

\subsubsection{\hcp supports the same protocols as \cp}
In \hcp, we can use \textsc{H-Mix} to compose two independent processes, but
once composed, we cannot form any direct connections between the two.
Unconnected process can block on each other's communications, and behave
differently due to each other's choices.
Therefore, processes in \hcp are strictly more expressive than processes in \cp
and parallel compositions of processes in \cp.
One example of a process which is expressible in \hcp, but not in \cp, is
$$
\seq[{
  \cpCase{x}{\piPar{P}{Q}}{\piPar{P'}{Q'}}
}]{
  \Gamma \hsep \tmty{x}{A \with B}, \Delta
}
$$
where $\seq[P]{\Gamma}$, $\seq[P']{\Gamma}$, $\seq[Q]{\tmty{x}{A}, \Delta}$ and
$\seq[Q]{\tmty{x}{B}, \Delta}$.

However, we can prove that \hcp inhabits the same types as \cp. This is a
property of the logic, and we will present it as such, omitting the process
terms. We prove this by showing that the hypersequent separator can be
interpreted as a tensor.\footnote{%
  In most hypersequent systems, the hypersequent separator is interpreted as a
  disjunctive connective. However, here we interpret it as a conjunctive
  connective.}
First, however, we prove a lemma showing that we can interpret the structural
comma as a par.
\begin{lemma}\label{lem:hcp-bigparr}
  If $\seq{\Gamma}$ in \hcp, then $\seq{\bigparr\Gamma}$ in \hcp.
\end{lemma}
\begin{proof}
  By induction on the structure of $\Gamma$.
  \begin{itemize}
  \item
    If $\ty{\Gamma} = \ty{A}$, then the statement is trivially true.
  \item
    If $\ty{\Gamma} = \ty{A, B, \Gamma'}$,
    then we apply $(\parr)$ to get $\seq{A \parr B, \Gamma'}$,\\
    and apply the induction hypothesis.
  \end{itemize}
\end{proof}

\begin{definition}\label{def:hcp-bigtens}
  We can interpret hyper-environments as types by collapsing them using a series
  of tensors. In the case of the empty hyper-environment
  $\ty{\emptyhypercontext}$, we interpret this as the unit of tensor.
  \begin{gather*}
    \begin{array}{lcl}
      \ty{\bigtens\emptyhypercontext}
      & = & \ty{\one}
      \\
      \ty{\bigtens(\Gamma_1 \hsep \dots \hsep \Gamma_n)}
      & = & \ty{\bigparr\Gamma_1 \tens \dots \tens \bigparr\Gamma_n}
    \end{array}
  \end{gather*}
\end{definition}
\begin{theorem}\label{thm:hcp-bigtens}
  If $\seq{\mathcal{G}}$ in \hcp, then $\seq{\bigtens\mathcal{G}}$ in \hcp.
\end{theorem}
\begin{proof}
  By case analysis on the structure of $\ty{\mathcal{G}}$.
  \begin{itemize}
  \item
    Case $\ty{\mathcal{G}} = \ty{\emptyhypercontext}$.\\
    By inversion on the derivation of $\seq{\emptyhypercontext}$.
    The only typing derivation ends in \textsc{H-Halt}.\\
    We rewrite as follows:
    \[
      \begin{array}{lcl}
        \AXC{}\NOM{H-Halt}\UIC{$\seq{\emptyhypercontext}$}
        \DisplayProof
        & \Rightarrow
        & \AXC{}\SYM{\one}\UIC{$\seq{\one}$}
          \DisplayProof
      \end{array}
    \] 
  \item
    Case $\ty{\mathcal{G}} = \ty{\Gamma_1 \hsep \dots \hsep \Gamma_n}$.\\
    By induction on the derivation of $\seq{\Gamma_1 \hsep \dots \hsep \Gamma_n}$.
    \begin{itemize}
    \item
      We remove any application of \textsc{H-Mix} where either premise is \textsc{H-Halt}.
    \item
      We propagate the applications of \textsc{H-Mix} downwards to the root of the
      proof, using the rewrite rules in \cref{fig:cc-hmix}, and obtain a
      derivation of the following form:
      \begin{prooftree}
        \pvar{$\rho_1$}\UIC{$\seq{\Gamma_1}$}
        \AXC{$\dots$}
        \pvar{$\rho_n$}\UIC{$\seq{\Gamma_n}$}
        \NOM{H-Mix}
        \TIC{$\seq{\Gamma_1 \hsep \dots \hsep \Gamma_n}$}
      \end{prooftree}
    \item
      We apply \cref{lem:hcp-bigparr} to each sub-proof $\rho_i$ to obtain a
      series of single-formula sequents, joined by applications of \textsc{H-Mix}.
    \item
      We replace the applications of \textsc{H-Mix} with applications of $(\tens)$.
    \end{itemize}
  \end{itemize}
\end{proof}
%%% Local Variables:
%%% TeX-master: "main"
%%% End:

As a corollary, we now know that \hcp inhabits the same types as \cp.
\newgeometry{right=0.75in,left=0.75in}
\begin{landscape}
\begin{figure*}[!p]
  \centering
  $$
  \begin{array}{lcl}
    \pvar{$\rho_1$}\UIC{$\seq{\mathcal{G}_1 \hsep {A}, \Gamma}$}
    \pvar{$\rho_2$}\UIC{$\seq{\mathcal{G}_2}$}
    \NOM{H-Mix}
    \BIC{$\seq{\mathcal{G}_1 \hsep \mathcal{G}_2}$}
    \pvar{$\rho_3$}\UIC{$\seq{\mathcal{G}_3 \hsep {A^\bot}, \Delta}$}
    \NOM{Cut}
    \BIC{$\seq{\mathcal{G}_1 \hsep \mathcal{G}_2 \hsep \mathcal{G}_3
    \hsep \Gamma, \Delta}$}
    \DisplayProof
    & \Longrightarrow
    & \pvar{$\rho_1$}\UIC{$\seq{\mathcal{G}_1 \hsep {A}, \Gamma}$}
      \pvar{$\rho_3$}\UIC{$\seq{\mathcal{G}_3 \hsep {A^\bot}, \Delta}$}
      \NOM{Cut}
      \BIC{$\seq{\mathcal{G}_1 \hsep \mathcal{G}_3 \hsep \Gamma, \Delta}$}
      \pvar{$\rho_2$}\UIC{$\seq{\mathcal{G}_2}$}
      \NOM{H-Mix}
      \BIC{$\seq{\mathcal{G}_1 \hsep \mathcal{G}_2 \hsep \mathcal{G}_3
      \hsep \Gamma, \Delta}$}
      \DisplayProof
    \\\\
    \pvar{$\rho_1$}\UIC{$\seq{\mathcal{G}_1 \hsep {A}, \Gamma}$}
    \pvar{$\rho_2$}\UIC{$\seq{\mathcal{G}_2}$}
    \NOM{H-Mix}
    \BIC{$\seq{\mathcal{G}_1 \hsep \mathcal{G}_2 \hsep {A}, \Gamma}$}
    \pvar{$\rho_3$}\UIC{$\seq{\mathcal{G}_3 \hsep {B}, \Delta}$}
    \SYM{\tens}
    \BIC{$\seq{\mathcal{G}_1 \hsep \mathcal{G}_2 \hsep \mathcal{G}_3
    \hsep {A \tens B}, \Gamma, \Delta}$}
    \DisplayProof
    & \Longrightarrow
    & \pvar{$\rho_1$}\UIC{$\seq{\mathcal{G}_1 \hsep {A}, \Gamma}$}
      \pvar{$\rho_3$}\UIC{$\seq{\mathcal{G}_3 \hsep {B}, \Delta}$}
      \SYM{\tens}
      \BIC{$\seq{\mathcal{G}_1 \hsep \mathcal{G}_2 \hsep {A \tens B},
      \Gamma, \Delta}$}
      \pvar{$\rho_2$}\UIC{$\seq{\mathcal{G}_2}$}
      \NOM{H-Mix}
      \BIC{$\seq{\mathcal{G}_1 \hsep \mathcal{G}_2 \hsep \mathcal{G}_3
      \hsep {A \tens B}, \Gamma, \Delta}$} 
      \DisplayProof
    \\\\
    \pvar{$\rho_1$}\UIC{$\seq{\mathcal{G}_1 \hsep {A}, {B}, \Gamma}$}
    \pvar{$\rho_2$}\UIC{$\seq{\mathcal{G}_2}$}
    \NOM{H-Mix}
    \BIC{$\seq{\mathcal{G}_1 \hsep \mathcal{G}_2 \hsep {A}, {B}, \Gamma}$}
    \SYM{\parr}
    \UIC{$\seq{\mathcal{G}_1 \hsep \mathcal{G}_2 \hsep {A \parr B}, \Gamma}$}
    \DisplayProof
    & \Longrightarrow
    & \pvar{$\rho_1$}\UIC{$\seq{\mathcal{G}_1 \hsep {A}, {B}, \Gamma}$}
      \SYM{\parr}
      \UIC{$\seq{\mathcal{G}_1 \hsep \mathcal{G}_2 \hsep {A}, {B}, \Gamma}$}
      \pvar{$\rho_2$}\UIC{$\seq{\mathcal{G}_2}$}
      \NOM{H-Mix}
      \BIC{$\seq{\mathcal{G}_1 \hsep \mathcal{G}_2 \hsep {A \parr B}, \Gamma}$}
      \DisplayProof
    \\\\
    \pvar{$\rho_1$}\UIC{$\seq{\mathcal{G}_1 \hsep \Gamma}$}
    \pvar{$\rho_2$}\UIC{$\seq{\mathcal{G}_2}$}
    \NOM{H-Mix}
    \BIC{$\seq{\mathcal{G}_1 \hsep \mathcal{G}_2 \hsep \Gamma}$}
    \SYM{\bot}
    \UIC{$\seq{\mathcal{G}_1 \hsep \mathcal{G}_2 \hsep {\bot}, \Gamma}$}
    \DisplayProof
    & \Longrightarrow
    & \pvar{$\rho_1$}\UIC{$\seq{\mathcal{G}_1 \hsep \Gamma}$}
      \SYM{\bot}
      \UIC{$\seq{\mathcal{G}_1 \hsep {\bot}, \Gamma}$}
      \pvar{$\rho_2$}\UIC{$\seq{\mathcal{G}_2}$}
      \NOM{H-Mix}
      \BIC{$\seq{\mathcal{G}_1 \hsep \mathcal{G}_2 \hsep \Gamma}$}
      \DisplayProof
    \\\\
    \pvar{$\rho_1$}\UIC{$\seq{\mathcal{G}_1 \hsep {A}, \Gamma}$}
    \pvar{$\rho_2$}\UIC{$\seq{\mathcal{G}_2}$}
    \NOM{H-Mix}
    \BIC{$\seq{\mathcal{G}_1 \hsep \mathcal{G}_2 \hsep {A}, \Gamma}$}
    \SYM{\plus_1}
    \UIC{$\seq{\mathcal{G}_1 \hsep \mathcal{G}_2 \hsep {A \plus B}, \Gamma}$}
    \DisplayProof
    & \Longrightarrow
    & \pvar{$\rho_1$}\UIC{$\seq{\mathcal{G}_1 \hsep {A}, \Gamma}$}
      \SYM{\plus_1}
      \UIC{$\seq{\mathcal{G}_1 \hsep {A \plus B}, \Gamma}$}
      \pvar{$\rho_2$}\UIC{$\seq{\mathcal{G}_2}$}
      \NOM{H-Mix}
      \BIC{$\seq{\mathcal{G}_1 \hsep \mathcal{G}_2 \hsep {A \plus B}, \Gamma}$}
      \DisplayProof
    \\\\
    \pvar{$\rho_1$}\UIC{$\seq{\mathcal{G}_1 \hsep {A}, \Gamma}$}
    \pvar{$\rho_2$}\UIC{$\seq{\mathcal{G}_2}$}
    \NOM{H-Mix}
    \BIC{$\seq{\mathcal{G}_1 \hsep \mathcal{G}_2 \hsep {A}, \Gamma}$}
    \pvar{$\rho_3$}\UIC{$\seq{\mathcal{G}_1 \hsep {B}, \Gamma}$}
    \pvar{$\rho_4$}\UIC{$\seq{\mathcal{G}_2}$} 
    \NOM{H-Mix}
    \BIC{$\seq{\mathcal{G}_1 \hsep \mathcal{G}_2 \hsep {B}, \Gamma}$}
    \SYM{\with}
    \BIC{$\seq{\mathcal{G}_1 \hsep \mathcal{G}_2 \hsep {A \with B}, \Gamma}$}
    \DisplayProof
    & \Longrightarrow
    & \pvar{$\rho_1$}\UIC{$\seq{\mathcal{G}_1 \hsep {A}, \Gamma}$}
      \pvar{$\rho_3$}\UIC{$\seq{\mathcal{G}_1 \hsep {B}, \Gamma}$}
      \SYM{\with}
      \BIC{$\seq{\mathcal{G}_1 \hsep {A \with B}, \Gamma}$}
      \pvar{$\rho_2$}\UIC{$\seq{\mathcal{G}_2}$}
      \NOM{H-Mix}
      \BIC{$\seq{\mathcal{G}_1 \hsep \mathcal{G}_2 \hsep {A \with B}, \Gamma}$}
      \DisplayProof
  \end{array}
  $$
  {\footnotesize
    (We have omitted the rewrite rules for the variants of \textsc{Cut} and $(\tens)$.)}
  \caption{Rewrite rules for \textsc{H-Mix} in \hcp.}
  \label{fig:cc-hmix}
\end{figure*}
\end{landscape}
\restoregeometry



\section{Hypersequent Cyclic Classical Processes}\label{sec:hccp}

\subsection{Terms}\label{sec:hccp-term}
\begin{definition}[Terms]\label{def:hccp-terms}
  \begin{gather*}
    \begin{aligned}
      \tm{P}, \tm{Q}, \tm{R}
           :=& \; \tm{\cpLink{x}{y}}         &&\text{link}
      \\ \mid& \; \tm{\piNew{x}{P}}          &&\text{name restriction, or ``new''}
      \\ \mid& \; \tm{( \piPar{P}{Q} )}      &&\text{parallel composition, or ``mix''}
      \\ \mid& \; \tm{\piHalt}               &&\text{terminated process}
      \\ \mid& \; \tm{\piBoundSend{x}{y}{P}} &&\text{output}
      \\ \mid& \; \tm{\piRecv{x}{y}{P}}      &&\text{input}
      \\ \mid& \; \tm{\piBoundSend{x}{}{P}}  &&\text{halt}
      \\ \mid& \; \tm{\cpWait{x}{}{P}}       &&\text{wait}
      \\ \mid& \; \tm{\cpInl{x}{P}}          &&\text{select left choice}
      \\ \mid& \; \tm{\cpInr{x}{P}}          &&\text{select right choice}
      \\ \mid& \; \tm{\cpCase{x}{P}{Q}}      &&\text{offer binary choice}
      \\ \mid& \; \tm{\cpAbsurd{x}}          &&\text{offer nullary choice}
    \end{aligned}
  \end{gather*}
\end{definition}
%%% Local Variables:
%%% TeX-master: "main"
%%% End:

\begin{definition}[Structural congruence]\label{def:hccp-equiv}
  We define the structural congruence $\equiv$ as the congruence closure over
  terms which satisfies the following additional axioms:
  \begin{gather*}
    \begin{array}{llcll}
      \cpEquivLinkComm
      & \tm{\cpLink{x}{y}}
      & \equiv
      & \tm{\cpLink{y}{x}}
      \\
      \hccpEquivMixComm
      & \tm{\piPar{P}{Q}}
      & \equiv
      & \tm{\piPar{Q}{P}}
      \\
      \hccpEquivMixAss1
      & \tm{\piPar{P}{( \piPar{Q}{R} )}}
      & \equiv
      & \tm{\piPar{( \piPar{P}{Q} )}{R}}
      \\
      \hccpEquivNewComm
      & \tm{\piNew{x}{\piNew{y}{P}}}
      & \equiv
      & \tm{\piNew{y}{\piNew{x}{P}}}
      \\
      \hccpEquivScopeExt1
      & \tm{\piNew{x}{( \piPar{P}{Q} )}}
      & \equiv
      & \tm{\piPar{P}{\piNew{x}{Q}}}
      & \text{if }\notFreeIn{x}{P} 
      \\
      \hccpEquivMixHalt1
      & \tm{\piPar{P}{\piHalt}}
      & \equiv
      & \tm{P}
    \end{array}
  \end{gather*}
\end{definition}
%%% Local Variables:
%%% TeX-master: "main"
%%% End:

\begin{definition}[Reduction]\label{def:ccp-reduction}
  A reduction $\reducesto{P}{Q}$ denotes that the process $\tm{P}$ can reduce
  to the process $\tm{Q}$ in a single step. Reductions can only be constructed
  as follows:
  \begin{gather*}
    \begin{array}{llcll}
      \cpRedAxCut1
      & \tm{\cpCut{x}{\cpLink{w}{x}}{P}}
      & \Longrightarrow
      & \tm{\cpSub{w}{x}{P}} 
      \\
      \cpRedBetaTensParr
      & \tm{\cpCut{x}{\piSend{x}{y}{P}}{\piRecv{x}{z}{R}}}
      & \Longrightarrow
      & \tm{\piNew{x}{\piNew{y}{(\piPar{P}{\cpSub{y}{z}{R}})}}}
      \\
      \cpRedBetaOneBot
      & \tm{\cpCut{x}{\piSend{x}{}{P}}{\piRecv{x}{}{P}}}
      & \Longrightarrow
      & \tm{\piPar{P}{Q}}
      \\
      \cpRedBetaPlusWith1
      & \tm{\cpCut{x}{\cpInl{x}{P}}{\cpCase{x}{Q}{R}}}
      & \Longrightarrow
      & \tm{\cpCut{x}{P}{Q}}
      \\
      \cpRedBetaPlusWith2
      & \tm{\cpCut{x}{\cpInr{x}{P}}{\cpCase{x}{Q}{R}}}
      & \Longrightarrow
      & \tm{\cpCut{x}{P}{R}}
    \end{array}
  \end{gather*}
  \begin{center}
    \begin{prooftree*}
      \AXC{$\reducesto{P}{P^\prime}$}
      \SYM{\hccpRedGammaNew}
      \UIC{$\reducesto{\piNew{x}{P}}{\piNew{x}{P^\prime}}$}
    \end{prooftree*}
    \begin{prooftree*}
      \AXC{$\reducesto{P}{P^\prime}$}
      \SYM{\hccpRedGammaMix}
      \UIC{$\reducesto{\piPar{P}{Q}}{\piPar{P^\prime}{Q}}$}
    \end{prooftree*}
    \begin{prooftree*}
      \AXC{$\tm{P}\equiv\tm{Q}$}
      \AXC{$\reducesto{Q}{Q^\prime}$}
      \AXC{$\tm{Q^\prime}\equiv\tm{P^\prime}$}
      \SYM{\hccpRedGammaEquiv}
      \TIC{$\reducesto{P}{P^\prime}$}
    \end{prooftree*}
  \end{center}
  The relations $\Longrightarrow^{?}$, $\Longrightarrow^{+}$, and
  $\Longrightarrow^\star$ are the reflexive, transitive and reflexive,
  transitive closures of $\Longrightarrow$. 
\end{definition}
%%% Local Variables:
%%% TeX-master: "main"
%%% End:


\subsection{Types}\label{sec:hccp-types}
We keep the types, environments, and hyper-environments from \hcp, see
\cref{def:cp-types,def:cp-envs,def:hcp-hyperenvs}.
\begin{definition}[Typing judgements]\label{def:hccp-typing}
  A typing judgement $\seq[P]{\Gamma_1 \hsep \dots \hsep \Gamma_n}$ denotes
  that the process $\tm{P}$ consists of $n$ independent, but potentially
  interleaved processes, each of which communicates according to its own
  protocol $\Gamma_i$. 
  Typing judgements can be constructed using the inference rules in
  \cref{fig:hccp}.
\end{definition}
\begin{figure*}[!htb]
  Structural Rules.
  \begin{center} \hccpInfAx        \hccpInfCycle \end{center}\vspace*{1\baselineskip}
  \begin{center} \hccpInfMix       \hccpInfHalt  \end{center}\vspace*{1\baselineskip}

  Logical Rules.
  \begin{center} \hccpInfBoundTens \hccpInfParr  \end{center}\vspace*{1\baselineskip}
  \begin{center} \hccpInfOne       \hccpInfBot   \end{center}\vspace*{1\baselineskip}
  \begin{center} \hccpInfPlus1     \hccpInfPlus2 \end{center}\vspace*{1\baselineskip}
  \begin{center} \hccpInfWith                    \end{center}\vspace*{1\baselineskip}
  \begin{center} \hccpInfNil       \hccpInfTop   \end{center}\vspace*{1\baselineskip}

  \centering
  (Where each logical rule has the side condition that $\tm{x} \not\in \ty{\mathcal{G}}$.)

  \caption{Hypersequent Cyclic Classical Processes with \textsc{H-Mix}}
  \label{fig:hccp}
\end{figure*}
%%% Local Variables:
%%% TeX-master: "main"
%%% End:


\subsection{Properties}\label{sec:hccp-properties}

\subsubsection{Preservation}\label{sec:hccp-preservation}
\begin{lemma}\label{lem:hccp-preservation-equiv}
  If $\tm{P}\equiv\tm{Q}$, then $\seq[P]{\mathcal{G}}$ iff $\seq[Q]{\mathcal{G}}$.
\end{lemma}
\begin{proof}
  By induction on the structure of $\tm{P}\equiv\tm{Q}$.
  \begin{itemize}
  \item
    Case \cpEquivLinkComm.
    We have to show $\seq[\cpLink{x}{y}]{\mathcal{G}} \Leftrightarrow
    \seq[\cpLink{y}{x}]{\mathcal{G}}$.
    For each direction, we continue by inversion on the typing derivation of the premise.
    The only typing derivation ends in \textsc{Ax}.
    For each direction, we rewrite as follows:
    \[
      \begin{array}{lcl}
        \AXC{}
        \NOM{Ax}
        \UIC{$\seq[\cpLink{x}{y}]{\tmty{x}{A}, \tmty{y}{A^\bot}}$}
        \DisplayProof
        & \Longrightarrow
        & \AXC{}
          \NOM{Ax}
          \UIC{$\seq[\cpLink{y}{x}]{\tmty{y}{A^\bot}, \tmty{x}{A^{\bot\bot}}}$}
          \NOM{\cref{lem:cp-negation-involutive}}
          \UIC{$\seq[\cpLink{y}{x}]{\tmty{y}{A^\bot}, \tmty{x}{A}}$}
          \DisplayProof
      \end{array}
    \]
  \item
    Case \hccpEquivMixComm.
    We have to show $\seq[(\piPar{P}{Q})]{\mathcal{G}} \Leftrightarrow
    \seq[(\piPar{Q}{P})]{\mathcal{G}}$.
    For each direction, we continue by inversion on the typing derivation of the
    premise.
    The only typing derivation ends in \textsc{H-Mix}.
    We rewrite as follows:
    \[
      \begin{array}{lcl}
        \AXC{$\seq[P]{\mathcal{G}}$}
        \AXC{$\seq[Q]{\mathcal{H}}$}
        \NOM{H-Mix}
        \BIC{$\seq[(\piPar{P}{Q})]{\mathcal{G} \hsep \mathcal{H}}$}
        \DisplayProof
        & \Leftrightarrow
        & \AXC{$\seq[Q]{\mathcal{H}}$}
          \AXC{$\seq[P]{\mathcal{G}}$}
          \NOM{H-Mix}
          \BIC{$\seq[(\piPar{Q}{P})]{\mathcal{H} \hsep \mathcal{G}}$}
          \DisplayProof
      \end{array}
    \]
  \item
    Case \hccpEquivMixAss1.
    We have to show $\seq[(\piPar{P}{(\piPar{Q}{R})})]{\mathcal{G}}
    \Leftrightarrow \seq[(\piPar{(\piPar{P}{Q})}{R})]{\mathcal{G}}$.
    For each direction, we continue by inversion on the typing derivation of the
    premise. 
    The only typing derivation ends in two applications of \textsc{H-Mix}.
    We rewrite as follows:
    \[
      \begin{array}{c}
        \AXC{$\seq[P]{\mathcal{G}_1}$}
        \AXC{$\seq[Q]{\mathcal{G}_2}$}
        \AXC{$\seq[R]{\mathcal{G}_3}$}
        \NOM{H-Mix}
        \BIC{$\seq[(\piPar{Q}{R})]{\mathcal{G}_2\hsep\mathcal{G}_3}$}
        \NOM{H-Mix}
        \BIC{$\seq[(\piPar{P}{(\piPar{Q}{R})})]{\mathcal{G}_1\hsep\mathcal{G}_2\hsep\mathcal{G}_3}$}
        \DisplayProof
        \\\\
        \Updownarrow
        \\\\
        \AXC{$\seq[P]{\mathcal{G}_1}$}
        \AXC{$\seq[Q]{\mathcal{G}_2}$}
        \NOM{H-Mix}
        \BIC{$\seq[(\piPar{P}{Q})]{\mathcal{G}_1\hsep\mathcal{G}_2}$}
        \AXC{$\seq[R]{\mathcal{G}_3}$}
        \NOM{H-Mix}
        \BIC{$\seq[(\piPar{P}{(\piPar{Q}{R})})]{\mathcal{G}_1\hsep\mathcal{G}_2\hsep\mathcal{G}_3}$}
        \DisplayProof
      \end{array}
    \]
  \item
    Case \hccpEquivNewComm.
    We have to show $\seq[\piNew{x}{\piNew{y}{P}}]{\mathcal{G}} \Leftrightarrow
    \seq[\piNew{y}{\piNew{x}{P}}]{\mathcal{G}}$.
    For each direction, we continue by inversion on the typing derivation of the premise.
    There are two typing derivations, each ends in two applications of \textsc{H-Cycle}.
    \begin{itemize}
    \item
      If $\tm{x}$ and $\tm{y}$ occur together in the same environment, we
      rewrite as follows:
      \[
        \begin{array}{c}
          \AXC{$\seq[P]{\mathcal{G} \hsep \tmty{x}{A}, \Gamma \hsep
          \tmty{x}{A^\bot}, \tmty{y}{B}, \Delta \hsep \tmty{y}{B^\bot},
          \Theta}$} 
          \NOM{H-Cycle}
          \UIC{$\seq[\piNew{y}{P}]{\mathcal{G} \hsep \tmty{x}{A}, \Gamma \hsep
          \tmty{x}{A^\bot}, \Delta, \Theta}$}
          \NOM{H-Cycle}
          \UIC{$\seq[\piNew{x}{\piNew{y}{P}}]{\mathcal{G} \hsep \Gamma, \Delta,
          \Theta}$} 
          \DisplayProof
          \\\\
          \Updownarrow
          \\\\
          \AXC{$\seq[P]{\mathcal{G} \hsep \tmty{x}{A}, \Gamma \hsep
          \tmty{x}{A^\bot}, \tmty{y}{B}, \Delta \hsep \tmty{y}{B^\bot},
          \Theta}$}
          \NOM{H-Cycle}
          \UIC{$\seq[\piNew{x}{P}]{\mathcal{G} \hsep \Gamma, \tmty{y}{B}, \Delta
          \hsep \tmty{y}{B^\bot}, \Theta}$}
          \NOM{H-Cycle}
          \UIC{$\seq[\piNew{x}{\piNew{y}{P}}]{\mathcal{G} \hsep \Gamma, \Delta,
          \Theta}$}
          \DisplayProof
        \end{array}
      \]
    \item
      Otherwise, we rewrite as follows:
      \[
        \begin{array}{c}
          \AXC{$\seq[P]{\mathcal{G} \hsep \tmty{x}{A}, \Gamma_1 \hsep
          \tmty{x}{A^\bot}, \Gamma_2 \hsep \tmty{y}{B}, \Delta_1 \hsep
          \tmty{y}{B^\bot}, \Delta_2}$}
          \NOM{H-Cycle}
          \UIC{$\seq[\piNew{y}{P}]{\mathcal{G} \hsep \tmty{x}{A}, \Gamma_1 \hsep
          \tmty{x}{A^\bot}, \Gamma_2 \hsep \Delta_1, \Delta_2}$} 
          \NOM{H-Cycle}
          \UIC{$\seq[\piNew{x}{\piNew{y}{P}}]{\mathcal{G} \hsep \Gamma_1,
          \Gamma_2 \hsep \Delta_1, \Delta_2}$}
          \DisplayProof
          \\\\
          \Updownarrow
          \\\\
          \AXC{$\seq[P]{\mathcal{G} \hsep \tmty{x}{A}, \Gamma_1 \hsep
          \tmty{x}{A^\bot}, \Gamma_2 \hsep \tmty{y}{B}, \Delta_1 \hsep
          \tmty{y}{B^\bot}, \Delta_2}$}
          \NOM{H-Cycle}
          \UIC{$\seq[\piNew{x}{P}]{\mathcal{G} \hsep \Gamma_1, \Gamma_2 \hsep
          \tmty{y}{B}, \Delta_1 \hsep \tmty{y}{B^\bot}, \Delta_2}$} 
          \NOM{H-Cycle}
          \UIC{$\seq[\piNew{x}{\piNew{y}{P}}]{\mathcal{G} \hsep \Gamma_1,
          \Gamma_2 \hsep \Delta_1, \Delta_2}$}
          \DisplayProof
        \end{array}
      \]
    \end{itemize}
  \item
    Case \hccpEquivScopeExt1.
    We have that $\notFreeIn{x}{P}$.
    We have to show $\seq[\piNew{x}{(\piPar{P}{Q})}]{\mathcal{G}}
    \Leftrightarrow \seq[(\piPar{P}{\piNew{x}{Q}})]{\mathcal{G}}$.
    For each direction, we continue by inversion on the typing derivation of the
    premise. 
    The only derivation ends in \textsc{H-Cycle} and \textsc{H-Mix}.
    We rewrite as follows:
    \[
      \begin{array}{c}
        \AXC{$\seq[P]{\mathcal{G}}$}
        \AXC{$\seq[Q]{\mathcal{H} \hsep \tmty{x}{A}, \Gamma \hsep
        \tmty{x}{A^\bot}, \Delta}$}
        \NOM{H-Mix}
        \BIC{$\seq[(\piPar{P}{Q})]{\mathcal{G} \hsep \mathcal{H} \hsep
        \tmty{x}{A}, \Gamma \hsep \tmty{x}{A^\bot}, \Delta}$} 
        \NOM{H-Cycle}
        \UIC{$\seq[\piNew{x}{(\piPar{P}{Q})}]{\mathcal{G} \hsep \mathcal{H}
        \hsep \Gamma, \Delta}$}
        \DisplayProof
        \\\\
        \Updownarrow
        \\\\
        \AXC{$\seq[P]{\mathcal{G}}$}
        \AXC{$\seq[Q]{\mathcal{H} \hsep \tmty{x}{A}, \Gamma \hsep
        \tmty{x}{A^\bot}, \Delta}$}
        \NOM{H-Cycle}
        \UIC{$\seq[\piNew{x}{Q}]{\mathcal{H} \hsep \tmty{x}{A}, \Gamma \hsep
        \tmty{x}{A^\bot}, \Delta}$}
        \NOM{H-Mix}
        \BIC{$\seq[(\piPar{P}{\piNew{x}{Q}})]{\mathcal{G} \hsep \mathcal{H}
        \hsep \Gamma, \Delta}$}
        \DisplayProof
      \end{array}
    \]
  \item
    Case \hccpEquivMixHalt1.
    We have to show that if $\seq[(\piPar{P}{\piHalt})]{\mathcal{G}}$, then
    $\seq[P]{\mathcal{G}}$, and the converse.
    In the $[\Rightarrow]$ direction, we proceed by inversion on the typing
    derivation of the premise.
    The only derivation ends in \textsc{H-Mix} and \textsc{H-Halt}.
    In the $[\Leftarrow]$ direction, we rewrite immediately.
    For either direction, we rewrite as follows:
    \[
      \begin{array}{lcl}
        \AXC{$\seq[P]{\mathcal{G}}$}
        \AXC{}
        \NOM{H-Halt}
        \UIC{$\seq[\piHalt]{\emptyhypercontext}$}
        \NOM{H-Mix}
        \BIC{$\seq[(\piPar{P}{\piHalt})]{\mathcal{G}}$}
        \DisplayProof
        & \Leftrightarrow
        & \AXC{$\seq[P]{\mathcal{G}}$}
          \DisplayProof
      \end{array}
    \]
  \end{itemize}
  The cases for reflexivity, transitivity, symmetry, and congruence are trivial.
\end{proof}
%%% Local Variables:
%%% TeX-master: "main"
%%% End:

\begin{lemma}\label{lem:hccp-preservation-subst}
  If $\seq[P]{\mathcal{G} \hsep \tmty{x}{A}, \Gamma}$,
  then $\seq[\cpSub{y}{x}{P}]{\mathcal{G} \hsep \tmty{y}{A}, \Gamma}$.
\end{lemma}
\begin{proof}
  By induction on the structure of the premise.
\end{proof}
%%% Local Variables:
%%% TeX-master: "main"
%%% End:

\begin{theorem}[Preservation]\label{thm:hccp-preservation}
  If $\reducesto{P}{Q}$ and $\seq[P]{\mathcal{G}}$, then $\seq[Q]{\mathcal{G}}$.
\end{theorem}
\begin{proof}
  By induction on the structure of $\reducesto{P}{Q}$.
  \begin{itemize}
  \item
    Case \hccpRedAxCut1.
    We have to show $\seq[\piNew{x}{(\piPar{\cpLink{w}{x}}{P})}]{\mathcal{G}}
    \Rightarrow \seq[\cpSub{w}{x}{P}]{\mathcal{G}}$.
    We continue by inversion on the typing derivation of the premise.
    The only typing derivation ends in \textsc{Cycle}, \textsc{Mix}, and
    \textsc{Ax}.
    We rewrite as follows:
    \[
      \begin{array}{c}
        \AXC{}
        \NOM{Ax}
        \UIC{$\seq[\cpLink{w}{x}]{\tmty{w}{A}, \tmty{x}{A^\bot}}$}
        \AXC{$\seq[P]{\mathcal{G} \hsep \tmty{x}{A}, \Gamma}$}
        \NOM{H-Mix}
        \BIC{$\seq[(\piPar{\cpLink{w}{x}}{P})]{\mathcal{G} \hsep \tmty{w}{A},
        \tmty{x}{A^\bot} \hsep \tmty{x}{A}, \Gamma}$}
        \NOM{H-Cycle}
        \UIC{$\seq[\piNew{x}{(\piPar{\cpLink{w}{x}}{P})}]{\mathcal{G} \hsep
        \tmty{w}{A}, \Gamma}$}
        \DisplayProof
        \\\\
        \Downarrow
        \\\\
        \AXC{$\seq[P]{\mathcal{G} \hsep \tmty{x}{A}, \Gamma}$}
        \NOM{\Cref{lem:hccp-preservation-subst}}
        \UIC{$\seq[\cpSub{w}{x}{P}]{\mathcal{G} \hsep \tmty{w}{A}, \Gamma}$}
        \DisplayProof
      \end{array}
    \]
  \item
    Case \hccpRedBetaTensParr.
    We have to show
    $\seq[\piNew{x}{(\piPar{\piBoundSend{x}{y}{P}}{\piRecv{x}{z}{Q}})}]{\mathcal{G}}
    \Rightarrow \seq[\piNew{x}{\piNew{y}{(\piPar{P}{\cpSub{y}{z}{Q}})}}]{\mathcal{G}}$.
    We continue by inversion on the typing derivation of the premise.
    The only typing derivation ends in \textsc{Cycle}, \textsc{Mix}, $(\tens)$,
    and $(\parr)$.
    We rewrite as follows: 
    \[
      \begin{array}{c}
        \AXC{$\seq[P]{
        \mathcal{G} \hsep \tmty{y}{A}, \Gamma \hsep \tmty{x}{B}, \Delta}$}
        \SYM{\tens}
        \UIC{$\seq[\piBoundSend{x}{y}{P}]{
        \mathcal{G} \hsep \tmty{x}{A \tens B}, \Gamma, \Delta}$}
        \AXC{$\seq[Q]{
        \mathcal{H} \hsep \tmty{z}{A^\bot}, \tmty{x}{B^\bot}, \Theta}$}
        \SYM{\parr}
        \UIC{$\seq[\piRecv{x}{z}{Q}]{
        \mathcal{H} \hsep \tmty{x}{A^\bot \parr B^\bot}, \Theta}$}
        \NOM{H-Mix}
        \BIC{$\seq[\piPar{\piBoundSend{x}{y}{P}}{\piRecv{x}{z}{Q}}]{
        \mathcal{G} \hsep \mathcal{H} \hsep
        \tmty{x}{A \tens B}, \Gamma, \Delta \hsep \tmty{x}{A^\bot \parr B^\bot}, \Theta}$}
        \NOM{H-Cycle}
        \UIC{$\seq[\piNew{x}{(\piPar{\piBoundSend{x}{y}{P}}{\piRecv{x}{z}{Q}})}]{
        \mathcal{G} \hsep \mathcal{H} \hsep \Gamma, \Delta, \Theta}$}
        \DisplayProof
        \\\\
        \Downarrow
        \\\\
        \AXC{$\seq[P]{
        \mathcal{G} \hsep \tmty{y}{A}, \Gamma \hsep \tmty{x}{B}, \Delta}$}
        \AXC{$\seq[Q]{
        \mathcal{H} \hsep \tmty{z}{A^\bot}, \tmty{x}{B^\bot}, \Theta}$}
        \NOM{\Cref{lem:hccp-preservation-subst}}
        \UIC{$\seq[\cpSub{y}{z}{Q}]{
        \mathcal{H} \hsep \tmty{y}{A^\bot}, \tmty{x}{B^\bot}, \Theta}$}
        \NOM{H-Mix}
        \BIC{$\seq[\piPar{P}{\cpSub{y}{z}{Q}}]{
        \mathcal{G} \hsep \mathcal{H} \hsep
        \tmty{y}{A}, \Gamma \hsep \tmty{x}{B}, \Delta \hsep
        \tmty{y}{A^\bot}, \tmty{x}{B^\bot}, \Theta}$}
        \NOM{H-Cycle}
        \UIC{$\seq[\piNew{y}{(\piPar{P}{\cpSub{y}{z}{Q}})}]{
        \mathcal{G} \hsep \mathcal{H} \hsep
        \tmty{x}{B}, \Delta \hsep \tmty{x}{B^\bot}, \Gamma, \Theta}$}
        \NOM{H-Cycle}
        \UIC{$\seq[\piNew{x}{\piNew{y}{(\piPar{P}{\cpSub{y}{z}{Q}})}}]{
        \mathcal{G} \hsep \mathcal{H} \hsep \Gamma, \Delta, \Theta}$}
        \DisplayProof
      \end{array}
    \]
  \item
    Case \hccpRedBetaOneBot.
    We have to show
    $\seq[\piNew{x}{(\piPar{\piSend{x}{}{P}}{\piRecv{x}{}{Q}})}]{\mathcal{G}}
    \Rightarrow \seq[(\piPar{P}{Q})]{\mathcal{G}}$.
    We continue by inversion on the typing derivation of the premise.
    The only typing derivation ends in \textsc{Cycle}, \textsc{Mix}, $(\one)$,
    and $(\bot)$.
    We rewrite as follows: 
    \[
      \begin{array}{c}
        \AXC{$\seq[P]{
        \mathcal{G}}$}
        \SYM{\one}
        \UIC{$\seq[\piSend{x}{}{P}]{
        \mathcal{G} \hsep \tmty{x}{\one}}$}
        \AXC{$\seq[Q]{
        \mathcal{H} \hsep \Delta}$}
        \SYM{\bot}
        \UIC{$\seq[\piRecv{x}{}{Q}]{
        \mathcal{H} \hsep \tmty{x}{\bot}, \Delta}$}
        \NOM{H-Mix}
        \BIC{$\seq[(\piPar{\piSend{x}{}{P}}{\piRecv{x}{}{Q}})]{
        \mathcal{G} \hsep \mathcal{H} \hsep \tmty{x}{\one} \hsep \tmty{x}{\bot}, \Delta}$}
        \NOM{H-Cycle}
        \UIC{$\seq[\piNew{x}{(\piPar{\piSend{x}{}{P}}{\piRecv{x}{}{Q}})}]{
        \mathcal{G} \hsep \mathcal{H} \hsep \Delta}$}
        \DisplayProof
        \\\\
        \Downarrow
        \\\\
        \AXC{$\seq[P]{\mathcal{G}}$}
        \AXC{$\seq[Q]{
        \mathcal{H} \hsep \Delta}$}
        \NOM{H-Mix}
        \BIC{$\seq[(\piPar{P}{Q})]{\mathcal{G} \hsep \mathcal{H} \hsep \Delta}$}
        \DisplayProof
      \end{array}
    \]
  \item
    Case \hccpRedBetaPlusWith1.
    We have to show
    $\seq[\piNew{x}{(\piPar{\cpInl{x}{P}}{\cpCase{x}{Q}{R}})}]{\mathcal{G}}
    \Rightarrow \seq[\piNew{x}{(\piPar{P}{Q})}]{\mathcal{G}}$.
    We continue by inversion on the typing derivation of the premise.
    The only typing derivation ends in \textsc{Cycle}, \textsc{Mix}, $(\plus_1)$,
    and $(\with)$.
    We rewrite as follows: 
    \[
      \begin{array}{c}
        \AXC{$\seq[P]{
        \mathcal{G} \hsep \tmty{x}{A}, \Gamma}$}
        \SYM{\plus_1}
        \UIC{$\seq[\cpInl{x}{P}]{
        \mathcal{G} \hsep \tmty{x}{A \plus B}, \Gamma}$}
        \AXC{$\seq[Q]{
        \mathcal{H} \hsep \tmty{x}{A^\bot}, \Delta}$}
        \AXC{$\seq[R]{
        \mathcal{H} \hsep \tmty{x}{B^\bot}, \Delta}$}
        \SYM{\with}
        \BIC{$\seq[\cpCase{x}{Q}{R}]{
        \mathcal{H} \hsep \tmty{x}{A^\bot \with B^\bot}, \Delta}$}
        \NOM{H-Mix} 
        \BIC{$\seq[(\piPar{\cpInl{x}{P}}{\cpCase{x}{Q}{R}})]{
        \mathcal{G} \hsep \mathcal{H} \hsep
        \tmty{x}{A \plus B}, \Gamma \hsep \tmty{x}{A^\bot \with B^\bot}, \Delta}$}
        \NOM{H-Cycle}
        \UIC{$\seq[\piNew{x}{(\piPar{\cpInl{x}{P}}{\cpCase{x}{Q}{R}})}]{
        \mathcal{G} \hsep \mathcal{H} \hsep \Gamma, \Delta}$}
        \DisplayProof
        \\\\
        \Downarrow
        \\\\
        \AXC{$\seq[P]{
        \mathcal{G} \hsep \tmty{x}{A}, \Gamma}$}
        \AXC{$\seq[Q]{
        \mathcal{H} \hsep \tmty{x}{A^\bot}, \Delta}$}
        \NOM{H-Mix}
        \BIC{$\seq[(\piPar{P}{Q})]{
        \mathcal{G} \hsep \mathcal{H} \hsep
        \tmty{x}{A}, \Gamma \hsep \tmty{x}{A^\bot}, \Delta}$}
        \NOM{H-Cycle}
        \UIC{$\seq[\piNew{x}{(\piPar{P}{Q})}]{
        \mathcal{G} \hsep \mathcal{H} \hsep \Gamma, \Delta}$}
        \DisplayProof
      \end{array}
    \]
  \item
    Case \hccpRedBetaPlusWith2.
    As above.
  \end{itemize}
\end{proof}
%%% Local Variables:
%%% TeX-master: "main"
%%% End:


\subsubsection{Canonical Forms and Progress}\label{sec:hccp-canonical-forms-and-progress}

\paragraph{Canonical Forms}\label{sec:hccp-canonical-forms}
\begin{definition}[Canonical forms]\label{def:hccp-canonical-forms}
  A process $\tm{P}$ is in canonical form if it is an action, or if it is of the
  form
  \[
    \tm{\piNew{x_1\dots x_n}{(P_1\ppar\dots\ppar P_{m+n+1})}}
  \]
  where each $\tm{P_i}$ is an action, no $\tm{P_i}$ is a link acting on a bound
  channel, and no two actions $\tm{P_i}$ and $\tm{P_j}$ act on the same bound
  channel.

  An immediate consequence of this definition is that if a process is in
  canonical form, then at least $m+1$ of the actions $\tm{P_i}$ are blocked on
  free channels.
\end{definition}
%%% Local Variables:
%%% TeX-master: "main"
%%% End:


\paragraph{Evaluation Contexts}\label{sec:hccp-evaluation-contexts}
\begin{definition}[Evaluation contexts]\label{def:hccp-evaluation-contexts}
  We define evaluation contexts as:
  \begin{gather*}
    \tm{E} := \tm{\hole} \mid \tm{\piNew{x}{E}} \mid \tm{(\piPar{E}{P})} \mid \tm{(\piPar{P}{E})}
  \end{gather*}
\end{definition}
\begin{definition}[Plugging]\label{def:hccp-evaluation-context-plugging}
  We define plugging for evaluation contexts as:
  \begin{gather*}
    \begin{array}{lcl}
      \tm{\cpPlug{\hole}{R}}          & := & \tm{R} \\
      \tm{\cpPlug{(\piNew{x}{E})}{R}} & := & \tm{\piNew{x}{\cpPlug{E}{R}}} \\
      \tm{\cpPlug{(\piPar{E}{P})}{R}} & := & \tm{\piPar{\cpPlug{E}{R}}{P}} \\
      \tm{\cpPlug{(\piPar{P}{E})}{R}} & := & \tm{\piPar{P}{\cpPlug{E}{R}}}
    \end{array}
  \end{gather*}
\end{definition}
%%% Local Variables:
%%% TeX-master: "main"
%%% End:

\begin{definition}[Evaluation prefixes]\label{def:hccp-evaluation-prefixes}
  We define evaluation prefixes as:
  \begin{align*}
    \tm{G}, \tm{H} := \tm{\hole} \mid \tm{\piNew{x}{G}} \mid \tm{(\piPar{G}{H})}
  \end{align*}
  The $\tm{\hole}$ construct represents a hole.
\end{definition}
\begin{definition}[Plugging]\label{def:hccp-evaluation-prefix-plugging}
  We define plugging for an evaluation prefix with $n$ holes as:
  \begin{gather*}
    \begin{array}{ll}
      \tm{\cpPlug{\hole}{R}} & := \; \tm{R} \\
      \tm{\cpPlug{\piNew{x}{G}}{R_1 \dots R_n}}
                            & := \; \tm{\piPar{x}{\cpPlug{G}{R_1 \dots R_m}}} \\
      \tm{\cpPlug{(\piPar{G}{H})}{R_1 \dots R_m, R_{m+1} \dots R_{n}}}
                            & := \; \tm{(\piPar{\cpPlug{G}{R_1 \dots R_m}}{\cpPlug{H}{R_{m+1} \dots R_n}})}
    \end{array}
  \end{gather*}
  Note that in the third case, $\tm{G}$ is an evaluation prefix with $m$ holes,
  and $\tm{H}$ is an evaluation prefix with $(n-m)$ holes. 
\end{definition}
%%% Local Variables:
%%% TeX-master: "main"
%%% End:

\begin{definition}[Maximum evaluation prefix]\label{def:hccp-maximum-evaluation-prefix}
  We say that $\tm{G}$ is the evaluation prefix of $\tm{P}$ when there exist terms
  $\tm{P_1} \dots \tm{P_n}$ such that $\tm{P} = \tm{\cpPlug{G}{P_1 \dots P_n}}$.
  We say that $\tm{G}$ is the maximum evaluation prefix if each $\tm{P_i}$ is an
  action. 
\end{definition}
%%% Local Variables:
%%% TeX-master: "main"
%%% End:

\input{lem-hccp-maximum-evaluation-prefixes}

\paragraph{Progress}\label{sec:hccp-progress}
\begin{theorem}[Progress]\label{thm:hccp-progress}
  If $\seq[P]{\mathcal{G}}$, then there exists a $\tm{Q}$ such that either
  $\tm{P}\equiv\tm{Q}$ and $\tm{Q}$ is in canonical form, or $\reducesto{P}{Q}$.
\end{theorem}
\begin{proof}
  By induction on the structure of the derivation for $\seq[P]{\mathcal{G}}$.
  The only interesting cases are when the last rule of the derivation is
  \textsc{H-Cycle} or \textsc{H-Mix}. In every other case, the typing rule
  constructs a term which is in canonical form. 

  If the last rule in the derivation is \textsc{H-Cycle} or \textsc{H-Mix}, we
  consider the maximum evaluation prefix $\tm{G}$ of $\tm{P}$.
  Let the prefix $\tm{G}$ consist of $n$ cycles, and $m$ mixes, which introduces
  $n$ channels, and composes $m+1$ actions.
  Furthermore, let $o$ denote the number of environments in the
  hyper-environment $\ty{\mathcal{G}}$.

  Actions are introduced by the logical rules, each of which is annotated with
  the side condition that to act on one end-point of a channel, the other
  end-point may not be in scope.
  Let us call a channel ``locked'' if both its end-points are in scope.
  In this sense, a mix can unlock \emph{one} channel introduced by a cycle,
  and $m$ mixes can unlock $m$ channels.
  As the hyper-environment $\ty{\mathcal{G}}$ consists of $o$ environments,
  there are $o - 1$ possible applications of \textsc{Mix} which do not unlock
  one of the $n$ channels introduced in $\tm{G}$.
  Let $n'$ refer to the number of unlocked channels.
  We have $m - (o - 1) \le n' \le m$.
  Therefore, one of the following must be true:
  \begin{itemize}
  \item
    One of the actions composed by $\tm{G}$ is a link $\tm{\cpLink{x}{y}}$
    acting on a bound channel.
    Suppose that $\tm{x}$ is the bound channel.
    There exist evaluation contexts $\tm{E}$, $\tm{E'}$, and $\tm{E''}$ such that
    \[
      \tm{P} =
      \tm{\cpPlug{E}{\piNew{x}{\cpPlug{E'}{\piPar{\cpPlug{E''}{\cpLink{x}{y}}}{R}}}}}.
    \]
    We rewrite by $\equiv$ to obtain
    \[
      \tm{P} \equiv
      \tm{\cpPlug{E}{\cpPlug{E'}{\cpPlug{E''}{\piNew{x}{( \piPar{\cpLink{x}{y}}{R} )}}}}}.
    \]
    We then reduce by applying \hccpRedAxCut1.
    Similarly if $\tm{y}$ is the bound channel.
  \item
    Two of the actions composed by $\tm{G}$ are acting on the \emph{same}
    bound channel.
    Let the two actions be $\tm{P_i}$ and $\tm{P_i}$, and the bound channel
    $\tm{x}$.
    There exist evaluation contexts $\tm{E}$, $\tm{E'}$, $\tm{E_i}$, and
    $\tm{E_j}$ such that
    \[
      \tm{P} =
      \tm{\cpPlug{E}{
          \piNew{x}{\cpPlug{E'}{
              \piPar{
                \cpPlug{E_i}{P_i}
              }{
                \cpPlug{E_j}{P_j}
              }}}}}
    \]
    We rewrite by $\equiv$ to obtain
    \[
      \tm{P} \equiv
      \tm{\cpPlug{E}{
          \cpPlug{E'}{
            \cpPlug{E_i}{
              \cpPlug{E_j}{
                \piNew{x}{(\piPar{P_i}{P_j})}
              }}}}}
    \]
    We then reduce by applying the appropriate \textbeta-reduction.
  \item
    Otherwise, (at least) one of the actions composed by $\tm{G}$ must be acting
    on a free channel.
    None of the actions composed by $\tm{G}$ is a link acting on a bound channel.
    No two actions composed by $\tm{G}$ are acting on the same bound channel.
    Therefore, $\tm{P}$ is equivalent to a process in canonical form.
  \end{itemize}
\end{proof}
%%% Local Variables:
%%% TeX-master: "main"
%%% End:


\subsection{Relation to \hcp}\label{sec:hcp2hccp}
\subsubsection{\hcp processes can be translated to \hccp}
The relation between \hccp and \hcp is slightly more complex than that between
\hcp and \cp.
In \cref{sec:hcp2cp}, we could simply insert terms in \cp into \hcp, as the term
language of \hcp is a simple extension of that of \cp.
In this case, however, we will have to start by defining a translation between
the two term languages.
\begin{definition}\label{def:hcp2hccp-terms}
  We define a translation from terms in \hcp to terms in \hccp:
  \begin{gather*}
    \begin{array}{lcl}
      \tm{\mtf{\cpLink{x}{y}}}
      & := & \tm{\cpLink{x}{y}} \\
      \tm{\mtf{\cpCut{x}{P}{Q}}}
      & := & \tm{\piNew{x}{(\piPar{\mtf{P}}{\mtf{Q}})}} \\
      \tm{\mtf{\cpSend{x}{y}{P}{Q}}}
      & := & \tm{\piBoundSend{x}{y}{(\piPar{\mtf{P}}{\mtf{Q}})}} \\
      \tm{\mtf{\cpRecv{x}{y}{P}}}
      & := & \tm{\piRecv{x}{y}{\mtf{P}}} \\
      \tm{\mtf{\cpHalt{x}}}
      & := & \tm{\piBoundSend{x}{}{\piHalt}} \\
      \tm{\mtf{\cpWait{x}{P}}}
      & := & \tm{\piRecv{x}{}{\mtf{P}}} \\
      \tm{\mtf{\cpInl{x}{P}}}
      & := & \tm{\cpInl{x}{\mtf{P}}} \\
      \tm{\mtf{\cpInr{x}{P}}}
      & := & \tm{\cpInr{x}{\mtf{P}}} \\
      \tm{\mtf{\cpCase{x}{P}{Q}}}
      & := & \tm{\cpCase{x}{\mtf{P}}{\mtf{Q}}} \\
      \tm{\mtf{\cpAbsurd{x}}}
      & := & \tm{\cpAbsurd{x}} \\
      \tm{\mtf{(\piPar{P}{Q})}}
      & := & \tm{(\piPar{\mtf{P}}{\mtf{Q}})}
    \end{array}
  \end{gather*}
  This translation breaks down the term constructs in \hcp into their more
  atomic constructs in \hccp, and rewrites bound send in terms of unbound send.
\end{definition}
%%% Local Variables:
%%% TeX-master: "main"
%%% End:

In the cases for $\tm{\cpCut{x}{P}{Q}}$, $\tm{\cpSend{x}{y}{P}{Q}}$ and
$\tm{\cpHalt{x}}$, while it does not look like much is happening, it is
important to realise that the constructs on the left-hand side are atomic,
whereas those on the right-hand side are complex. 

We can show that this translation respects typing.
\begin{theorem}\label{thm:hcp2hccp-typing}
  If $\seq[P]{\mathcal{G}}$ in \hcp, then $\seq[\mtf{P}]{\mathcal{G}}$ in \hccp.
\end{theorem}
\begin{proof}
  By induction on the structure of the derivation of $\seq[P]{\mathcal{G}}$.
  \begin{itemize}
  \item 
    Case \textsc{Cut}.
    We rewrite as follows:
    \[
      \begin{array}{c}
        \AXC{$\seq[P]{\mathcal{G} \hsep \tmty{x}{A}, \Gamma}$}
        \AXC{$\seq[Q]{\mathcal{H} \hsep \tmty{x}{A^\bot}, \Delta}$}
        \NOM{Cut}
        \BIC{$\seq[\cpCut{x}{P}{Q}]{
        \mathcal{G} \hsep \mathcal{H} \hsep \Gamma, \Delta}$}
        \DisplayProof
        \\\\
        \Downarrow
        \\\\
        \AXC{$\seq[\mtf{P}]{\mathcal{G} \hsep \tmty{x}{A}, \Gamma}$}
        \AXC{$\seq[\mtf{Q}]{\mathcal{H} \hsep \tmty{x}{A^\bot}, \Delta}$}
        \NOM{H-Mix} 
        \BIC{$\seq[\piPar{\mtf{P}}{\mtf{Q}}]{
        \mathcal{G} \hsep \mathcal{H} \hsep
        \tmty{x}{A}, \Gamma \hsep \tmty{x}{A^\bot}, \Delta}$}
        \NOM{H-Cycle}
        \UIC{$\seq[\piNew{x}{(\piPar{\mtf{P}}{\mtf{Q}})}]{
        \mathcal{G} \hsep \mathcal{H} \hsep \Gamma, \Delta}$}
        \DisplayProof
      \end{array}
    \]
  \item
    Case $(\tens)$.
    We rewrite as follows:
    \[
      \begin{array}{c}
        \AXC{$\seq[P]{\mathcal{G} \hsep \tmty{y}{A}, \Gamma}$}
        \AXC{$\seq[Q]{\mathcal{H} \hsep \tmty{x}{B}, \Delta}$}
        \SYM{\tens}
        \BIC{$\seq[\cpSend{x}{y}{P}{Q}]{
        \mathcal{G} \hsep \mathcal{H} \hsep \tmty{x}{A \tens B}, \Gamma, \Delta}$}
        \DisplayProof
        \\\\
        \Downarrow
        \\\\
        \AXC{$\seq[\mtf{P}]{\mathcal{G} \hsep \tmty{y}{A}, \Gamma}$}
        \AXC{$\seq[\mtf{Q}]{\mathcal{H} \hsep \tmty{x}{B}, \Delta}$}
        \NOM{H-Mix}
        \BIC{$\seq[\piPar{\mtf{P}}{\mtf{Q}}]{
        \mathcal{G} \hsep \mathcal{H} \hsep
        \tmty{y}{A}, \Gamma \hsep \tmty{x}{B}, \Delta}$}
        \NOM{H-Cycle}
        \UIC{$\seq[\piNew{x}{(\piPar{\mtf{P}}{\mtf{Q}})}]{
        \mathcal{G} \hsep \mathcal{H} \hsep \Gamma, \Delta}$}
        \DisplayProof
      \end{array}
    \]
  \item
    Case $(\one)$.
    We rewrite as follows:
    \[
      \begin{array}{lcl}
        \AXC{$\seq[\cpHalt{x}]{\tmty{x}{\one}}$}
        \DisplayProof
        & \Rightarrow
        & \AXC{}
          \NOM{H-Halt}
          \UIC{$\seq[\piHalt]{\emptyhypercontext}$}
          \SYM{\one}
          \UIC{$\seq[\piBoundSend{x}{}{\piHalt}]{\tmty{x}{\one}}$}
          \DisplayProof
      \end{array}      
    \]
  \end{itemize}
  The other cases are trivial.
\end{proof}
%%% Local Variables:
%%% TeX-master: "main"
%%% End:


We can show that this translation respects the structural congruence---that is
to say, if two terms are equivalent in \hcp, then their translations are
equivalent in \hccp.
\begin{theorem}\label{thm:hcp2hccp-equiv}
  If $\tm{P}\equiv\tm{Q}$ in \hcp, then $\tm{\mtf{P}}\equiv\tm{\mtf{Q}}$ in \hccp.
\end{theorem}
\begin{proof}
  By induction on the structure of the derivation of $\tm{P}\equiv\tm{Q}$.
  \begin{itemize}
  \item
    Case \hcpEquivCutComm.
    \begin{gather*}
      \begin{array}{lcl}
        \tm{\piNew{x}{(\piPar{\mtf{P}}{\mtf{Q}})}}
        & \equiv & \text{by \hccpEquivMixComm}
        \\
        \tm{\piNew{x}{(\piPar{\mtf{Q}}{\mtf{P}})}}
      \end{array}
    \end{gather*}
  \item
    Case \hcpEquivCutAss1.
    \begin{gather*}
      \begin{array}{lcl}
        \tm{\piNew{x}{(\piPar{\mtf{P}}{\piNew{y}{(\piPar{\mtf{Q}}{\mtf{R}})}})}}
        & \equiv & \text{by \hccpEquivScopeExt2}
        \\
        \tm{\piNew{x}{\piNew{y}{(\piPar{\mtf{P}}{(\piPar{\mtf{Q}}{\mtf{R}})})}}}
        & \equiv & \text{by \hccpEquivNewComm}
        \\
        \tm{\piNew{y}{\piNew{x}{(\piPar{\mtf{P}}{(\piPar{\mtf{Q}}{\mtf{R}})})}}}
        & \equiv & \text{by \hccpEquivMixAss1}
        \\
        \tm{\piNew{y}{\piNew{x}{(\piPar{(\piPar{\mtf{P}}{\mtf{Q}})}{\mtf{R}})}}}
        & \equiv & \text{by \hccpEquivScopeExt1 and \hccpEquivMixComm}
        \\
        \tm{\piNew{y}{(\piPar{\piNew{x}{(\piPar{\mtf{P}}{\mtf{Q}})}}{\mtf{R}})}}
      \end{array}
    \end{gather*}
  \item
    Case \hcpEquivMixCut1.
    \begin{gather*}
      \begin{array}{lcl}
        \tm{\piNew{x}{(\piPar{(\piPar{P}{Q})}{R})}}
        & \equiv & \text{by \hccpEquivMixAss2}
        \\
        \tm{\piNew{x}{(\piPar{P}{(\piPar{Q}{R})})}}
        & \equiv & \text{by \hccpEquivScopeExt1}
        \\
        \tm{(\piPar{P}{\piNew{x}{(\piPar{Q}{R})}})}
      \end{array}
    \end{gather*}
  \end{itemize}
  The other cases are trivial.
\end{proof}
%%% Local Variables:
%%% TeX-master: "main"
%%% End:


Furthermore, we can show that this translation respects reductions.
\begin{theorem}\label{thm:hcp2hccp-reduction}
  If $\reducesto{P}{Q}$ in \hcp, then $\reducesto{\mtf{P}}{\mtf{Q}}$ in \hccp.
\end{theorem}
\begin{proof}
  By induction on the structure of the derivation of $\reducesto{P}{Q}$.
  \begin{itemize}
  \item
    Case \cpRedBetaTensParr.
    \begin{gather*}
      \begin{array}{lcl}
        \tm{\piNew{x}{(\piPar{\piBoundSend{x}{y}{(\piPar{P}{Q})}}{\piRecv{x}{z}{R}})}}
        & \Longrightarrow & \text{by \hccpRedBetaTensParr}
        \\
        \tm{\piNew{x}{\piNew{y}{(\piPar{(\piPar{P}{Q})}{\cpSub{y}{z}{R}})}}}
        & \equiv & \text{by \hccpEquivNewComm}
        \\
        \tm{\piNew{y}{\piNew{x}{(\piPar{(\piPar{P}{Q})}{\cpSub{y}{z}{R}})}}}
        & \equiv & \text{by \hccpEquivMixAss1}
        \\
        \tm{\piNew{y}{\piNew{x}{(\piPar{P}{(\piPar{Q}{\cpSub{y}{z}{R}})})}}}
        & \equiv & \text{by \hccpEquivScopeExt1}
        \\
        \tm{\piNew{y}{(\piPar{P}{\piNew{x}{(\piPar{Q}{\cpSub{y}{z}{R}})}})}}
      \end{array}
    \end{gather*}
  \item
    Case \cpRedBetaOneBot.
    \begin{gather*}
      \begin{array}{lcl}
        \tm{\piNew{x}{(\piPar{\piBoundSend{x}{}{\piHalt}}{\piRecv{x}{}{\mtf{Q}}})}}
        & \Longrightarrow & \text{by \hccpRedBetaOneBot}
        \\
        \tm{(\piPar{\piHalt}{\mtf{Q}})}
        & \equiv & \text{by \hccpEquivMixHalt1}
        \\
        \tm{\mtf{Q}}
      \end{array}
    \end{gather*}
  \end{itemize}
  The other cases are trivial.
\end{proof}
%%% Local Variables:
%%% TeX-master: "main"
%%% End:

\begin{theorem}
  If $\reducesto{\mtf{P}}{Q}$ in \hccp, then there exists an $\tm{R}$ such that
  $\tm{\mtf{R}}\equiv\tm{Q}$ and $\reducesto{P}{R}$ in \hcp.
\end{theorem}
\begin{proof}
  By induction on the structure of the derivation of $\reducesto{\mtf{P}}{Q}$.
  \begin{itemize}
  \item 
    Case $\hccpRedBetaTensParr$.\\
    We have
    $\tm{P}\equiv\tm{\cpCut{x}{\piBoundSend{x}{y}{(
          \piPar{P'}{Q'})}}{\piRecv{x}{z}{R'}}}$ and
    $\tm{Q}\equiv\tm{\piNew{x}{\piNew{y}{(\piPar{(
            \piPar{\mtf{P'}}{\mtf{Q'}})}{\cpSub{y}{z}{\mtf{R'}}})}}}$. 

    We let $\tm{R} = \tm{\cpCut{y}{P'}{\cpCut{x}{Q'}{\cpSub{y}{z}{R'}}}}$.

    We have
    \begin{gather*}
      \setlength{\arraycolsep}{1pt}
      \begin{array}{lcl}
        \tm{\piNew{y}{(\piPar{\mtf{P'}}{\piNew{x}{(
        \piPar{\mtf{Q'}}{\mtf{\cpSub{y}{z}{R'}}})}})}}
        & \equiv & \text{by \hccpEquivScopeExt1}
        \\
        \tm{\piNew{y}{\piNew{x}{(\piPar{\mtf{P'}}{(
        \piPar{\mtf{Q'}}{\mtf{\cpSub{y}{z}{R'}}})})}}}
        & \equiv & \text{by \hccpEquivMixAss1}
        \\
        \tm{\piNew{y}{\piNew{x}{(\piPar{(
        \piPar{\mtf{P'}}{\mtf{Q'}})}{\mtf{\cpSub{y}{z}{R'}}})}}}
        & \equiv & \text{by \hccpEquivNewComm}
        \\
        \tm{\piNew{x}{\piNew{y}{(\piPar{(
        \piPar{\mtf{P'}}{\mtf{Q'}})}{\mtf{\cpSub{y}{z}{R'}}})}}}
      \end{array}
    \end{gather*}
    and
    \begin{gather*}
      \setlength{\arraycolsep}{1pt}
      \begin{array}{lcl}
        \tm{\cpCut{x}{\piBoundSend{x}{y}{(\piPar{P'}{Q'})}}{\piRecv{x}{z}{R'}}}
        & \Longrightarrow & \text{by \hccpRedBetaTensParr}
        \\
        \tm{\cpCut{y}{P'}{\cpCut{x}{Q'}{\cpSub{y}{z}{R'}}}}.
      \end{array}
    \end{gather*}
  \item
    Case $\hccpRedBetaOneBot$.\\
    We have $\tm{P} \equiv
    \tm{\piNew{x}{(\piPar{\piBoundSend{x}{}{\piHalt}}{\piRecv{x}{}{Q'}})}}$,
    and $\tm{Q} \equiv \tm{(\piPar{\piHalt}{Q'})}$.

    We let $\tm{R} = \tm{Q'}$.

    We have $\tm{\mtf{Q'}} \equiv \tm{(\piPar{\piHalt}{\mtf{Q'}})}$ by
    $\hccpEquivMixHalt1$ and
    $\tm{\piNew{x}{(\piPar{\piBoundSend{x}{}{\piHalt}}{\piRecv{x}{}{Q'}})}}
    \Longrightarrow \tm{Q'}$ by $\cpRedBetaOneBot$.
  \end{itemize}
  The other cases are trivial.
\end{proof}
%%% Local Variables:
%%% TeX-master: "main"
%%% End:


\subsubsection{\hccp supports the same protocols as \hcp}\label{sec:hccp2hcp}
There is no neat translation from \hccp to \hcp, as there is no way to translate
processes of the form $\tm{\piSend{x}{}{P}}$ in \hcp---though there is probably
a way to simulate them.
Instead, similarly to \cref{sec:hcp2cp}, we will prove that \hccp inhabits the
same sequents as \hcp.
We prove this by showing that each proof in \hccp can be transformed into a
proof in \hcp.
The reduction behaviour between the two processes associated with these proofs,
however, may differ.
Therefore, we will present this as a property of the logic, and we will present
it as such, omitting the process terms.
\begin{theorem}\label{thm:hccp2hcp}
  If $\seq{\mathcal{G}}$ in \hccp, then $\seq{\mathcal{G}}$ in \hcp.
\end{theorem}
\begin{proof}
  By induction on the structure of the derivation of $\seq{\mathcal{G}}$.
  \begin{itemize}
  \item 
    Case \textsc{H-Cycle}.\\
    Each application of \textsc{H-Cycle} introduces one hypersequent separator,
    which can only be eliminated by \textsc{H-Mix}.
    Therefore, under each application of \textsc{H-Cycle}, there must be a
    corresponding application of \textsc{H-Mix}.
    We rewrite as follows:
    \[
      \begin{array}{lcl}
        \pvar{$\rho_1$}\UIC{$\seq{\mathcal{G}_1 \hsep \ty{A}, \Gamma}$}
        \pvar{$\rho_2$}\UIC{$\seq{\mathcal{G}_2 \hsep \ty{A^\bot}, \Delta}$}
        \NOM{H-MIX}
        \BIC{$\seq{\mathcal{G}_1 \hsep \mathcal{G}_2 \hsep
        \ty{A}, \Gamma \hsep \ty{A^\bot}, \Delta}$}
        \noLine\UIC{$\vphantom{\Gamma}\smash[t]{\vdots}$}
        \noLine\UIC{$\rho_3$}
        \noLine\UIC{$\vphantom{\Gamma}\smash[t]{\vdots}$}
        \noLine\UIC{$\seq{\mathcal{G} \hsep \ty{A}, \Gamma' \hsep \ty{A^\bot}, \Delta'}$}
        \NOM{H-Cycle}
        \UIC{$\seq{\mathcal{G} \hsep \Gamma', \Delta'}$}
        \DisplayProof
        & \Rightarrow
        & \pvar{$\rho_1$}\UIC{$\seq{\mathcal{G}_1 \hsep \ty{A}, \Gamma}$}
          \pvar{$\rho_2$}\UIC{$\seq{\mathcal{G}_2 \hsep \ty{A^\bot}, \Delta}$}
          \NOM{Cut}
          \BIC{$\seq{\mathcal{G}_1 \hsep \mathcal{G}_2 \hsep \Gamma, \Delta}$}
          \noLine\UIC{$\vphantom{\Gamma}\smash[t]{\vdots}$}
          \noLine\UIC{$\rho_3$}
          \noLine\UIC{$\vphantom{\Gamma}\smash[t]{\vdots}$}
          \noLine\UIC{$\seq{\mathcal{G} \hsep \Gamma', \Delta'}$}
          \DisplayProof
      \end{array}
    \]
  \item
    Case $(\tens)$.\\
    We rewrite as follows:
    \[
      \begin{array}{lcl}
        \pvar{$\rho_1$}\UIC{$\seq{\mathcal{G}_1 \hsep \ty{A}, \Gamma}$}
        \pvar{$\rho_2$}\UIC{$\seq{\mathcal{G}_2 \hsep \ty{B}, \Delta}$}
        \NOM{H-MIX}
        \BIC{$\seq{\mathcal{G}_1 \hsep \mathcal{G}_2 \hsep
        \ty{A}, \Gamma \hsep \ty{B}, \Delta}$}
        \noLine\UIC{$\vphantom{\Gamma}\smash[t]{\vdots}$}
        \noLine\UIC{$\rho_3$}
        \noLine\UIC{$\vphantom{\Gamma}\smash[t]{\vdots}$}
        \noLine\UIC{$\seq{\mathcal{G} \hsep \ty{A}, \Gamma' \hsep \ty{B}, \Delta'}$}
        \SYM{\tens}
        \UIC{$\seq{\mathcal{G} \hsep \ty{A \tens B}, \Gamma', \Delta'}$}
        \DisplayProof
        & \Rightarrow
        & \pvar{$\rho_1$}\UIC{$\seq{\mathcal{G}_1 \hsep \ty{A}, \Gamma}$}
          \pvar{$\rho_2$}\UIC{$\seq{\mathcal{G}_2 \hsep \ty{B}, \Delta}$}
          \SYM{\tens}
          \BIC{$\seq{\mathcal{G}_1 \hsep \mathcal{G}_2 \hsep
          \ty{A \tens B}, \Gamma, \Delta}$}
          \noLine\UIC{$\vphantom{\Gamma}\smash[t]{\vdots}$}
          \noLine\UIC{$\rho_3$}
          \noLine\UIC{$\vphantom{\Gamma}\smash[t]{\vdots}$}
          \noLine\UIC{$\seq{\mathcal{G} \hsep \ty{A \tens B}, \Gamma', \Delta'}$}
          \DisplayProof
      \end{array}
    \]
  \item
    Case $(\one)$.\\
    We rewrite as follows:
    \[
      \begin{array}{lcl}
        \pvar{$\rho$}\UIC{$\seq{\mathcal{G}}$}
        \SYM{\one}
        \UIC{$\seq{\mathcal{G} \hsep \ty{\one}}$}
        \DisplayProof
        & \Rightarrow
        & \pvar{$\rho$}\UIC{$\seq{\mathcal{G}}$}
          \AXC{}\SYM{\one}\UIC{$\seq{\ty{\one}}$}
          \NOM{H-Mix}
          \BIC{$\seq{\mathcal{G} \hsep \ty{\one}}$}
          \DisplayProof 
      \end{array}
    \]
  \end{itemize}
  The other cases are trivial.
\end{proof}
%%% Local Variables:
%%% TeX-master: "main"
%%% End:

As a corollary, we now have that \hccp inhabits the same types as \cp.

\section{Priority-based Classical Processes}\label{sec:ccp}
The type system for \cp with cyclic communication structures (\ccp) is described
by \textcite{dardha2018}.
It is strictly more expressive than \cp and \hccp, as it allows cyclic
communication structures to be formed.
It enforces deadlock freedom by checking that the communication graph is
acyclic.
It uses the same definitions for terms as \hccp, and the following definitions
for structural congruence and term reduction. 
\begin{definition}[Structural congruence]\label{def:ccp-equiv}
  We define the structural congruence $\equiv$ as the congruence closure over
  terms which satisfies the following additional axioms:
  \begin{gather*}
    \begin{array}{llcll}
      \cpEquivLinkComm
      & \tm{\cpLink{x}{y}}
      & \equiv
      & \tm{\cpLink{y}{x}}
      \\
      \ccpEquivLinkCut
      & \tm{\piNew{xy}{\cpLink{x}{y}}}
      & \equiv
      & \tm{\piHalt}
      \\
      \hccpEquivMixComm
      & \tm{\piPar{P}{Q}}
      & \equiv
      & \tm{\piPar{Q}{P}}
      \\
      \hccpEquivMixAss1
      & \tm{\piPar{P}{( \piPar{Q}{R} )}}
      & \equiv
      & \tm{\piPar{( \piPar{P}{Q} )}{R}}
      \\
      \hccpEquivNewComm
      & \tm{\piNew{xy}{\piNew{zw}{P}}}
      & \equiv
      & \tm{\piNew{zw}{\piNew{xy}{P}}}
      \\
      \hccpEquivScopeExt1
      & \tm{\piNew{xy}{( \piPar{P}{Q} )}}
      & \equiv
      & \tm{\piPar{P}{\piNew{xy}{Q}}}
      & \text{if }\tm{x},\tm{y}\not\in\tm{P}
      \\
      \hccpEquivMixHalt1
      & \tm{\piPar{P}{\piHalt}}
      & \equiv
      & \tm{P}
    \end{array}
  \end{gather*}
\end{definition}
%%% Local Variables:
%%% TeX-master: "main"
%%% End:

\begin{definition}[Reduction]\label{def:hccp-reduction}
  A reduction $\reducesto{P}{Q}$ denotes that the process $\tm{P}$ can reduce
  to the process $\tm{Q}$ in a single step. Reductions can only be constructed
  as follows:
  \begin{gather*}
    \begin{array}{llcll}
      \cpRedAxCut1
      & \tm{\cpCut{xy}{\cpLink{w}{x}}{P}}
      & \Longrightarrow
      & \tm{\cpSub{w}{y}{P}}
      \\
      \cpRedBetaTensParr
      & \tm{\cpCut{xy}{\piBoundSend{x}{z}{P}}{\piRecv{y}{w}{R}}}
      & \Longrightarrow
      & \tm{\piNew{xy}{\piNew{zw}{(\piPar{P}{R})}}}
      \\
      \cpRedBetaOneBot
      & \tm{\cpCut{xy}{\piBoundSend{x}{}{P}}{\piRecv{y}{}{P}}}
      & \Longrightarrow
      & \tm{\piPar{P}{Q}}
      \\
      \cpRedBetaPlusWith1
      & \tm{\cpCut{xy}{\cpInl{x}{P}}{\cpCase{y}{Q}{R}}}
      & \Longrightarrow
      & \tm{\cpCut{xy}{P}{Q}}
      \\
      \cpRedBetaPlusWith2
      & \tm{\cpCut{xy}{\cpInr{x}{P}}{\cpCase{y}{Q}{R}}}
      & \Longrightarrow
      & \tm{\cpCut{xy}{P}{R}}
    \end{array}
  \end{gather*}
  \begin{center}
    \begin{prooftree*}
      \AXC{$\reducesto{P}{P^\prime}$}
      \SYM{\hccpRedGammaNew}
      \UIC{$\reducesto{\piNew{xy}{P}}{\piNew{xy}{P^\prime}}$}
    \end{prooftree*}
    \begin{prooftree*}
      \AXC{$\reducesto{P}{P^\prime}$}
      \SYM{\hccpRedGammaMix}
      \UIC{$\reducesto{\piPar{P}{Q}}{\piPar{P^\prime}{Q}}$}
    \end{prooftree*}
  \end{center}
  \begin{prooftree}
    \AXC{$\tm{P}\equiv\tm{Q}$}
    \AXC{$\reducesto{Q}{Q^\prime}$}
    \AXC{$\tm{Q^\prime}\equiv\tm{P^\prime}$}
    \SYM{\hccpRedGammaEquiv}
    \TIC{$\reducesto{P}{P^\prime}$}
  \end{prooftree}
  The relations $\Longrightarrow^{?}$, $\Longrightarrow^{+}$, and
  $\Longrightarrow^\star$ are the reflexive, the transitive, and the reflexive,
  transitive closures of $\Longrightarrow$.
\end{definition}
%%% Local Variables:
%%% TeX-master: "main"
%%% End:

It uses the same types and environments as \hccp, though it annotates the
connectives of types with \emph{priorities}, e.g.\ as $\ty{A \tens^{\cs{o}} B}$.
A priority can be thought of as a label for an action.
Priorities form a partial order, where $\cs{o}<\cs{o'}$ means that the action
labeled with $\cs{o}$ must happen before the action labeled with $\cs{o'}$.
In \textcite{dardha2018}, priorities are natural numbers.
\begin{definition}[Priority function]\label{def:ccp-pr}
  We define the function $\pr{\cdot}$ to extract the top-level priority from types:
  \begin{gather*}
    \begin{array}{lclclcl}
      {}\pr{\ty{\one^{\cs{o}}}}      & = & \cs{o} & \qquad
      & \pr{\ty{\nil^{\cs{o}}}}      & = & \cs{o} \\
      {}\pr{\ty{\bot^{\cs{o}}}}      & = & \cs{o} & \qquad
      & \pr{\ty{\top^{\cs{o}}}}      & = & \cs{o} \\
      {}\pr{\ty{A \tens^{\cs{o}} B}} & = & \cs{o} & \qquad
      & \pr{\ty{A \plus^{\cs{o}} B}} & = & \cs{o} \\
      {}\pr{\ty{A \parr^{\cs{o}} B}} & = & \cs{o} & \qquad
      & \pr{\ty{A \with^{\cs{o}} B}} & = & \cs{o}
    \end{array}
  \end{gather*}
  We extend it to environments to extract the smallest priority:
  \begin{gather*}
    \pr{\tmty{x_1}{A_1}, \dots, \tmty{x_n}{A_n}} = \cs{\bigwedge_{i=1}^{n}\pr{A_i}}
  \end{gather*} 
\end{definition}
%%% Local Variables:
%%% TeX-master: "main"
%%% End:

\begin{definition}[Typing judgements]\label{def:ccp-typing}
  A typing judgement $\seq[P]{\tmty{x_1}{A_1}\dots\tmty{x_n}{A_n}}$ denotes that
  the process $\tm{P}$ communicates along channels $\tm{x_1}\dots\tm{x_n}$
  following protocols $\ty{A_1}\dots\ty{A_n}$.
  Typing judgements can be constructed using the inference rules in
  \cref{fig:ccp}.
\end{definition}
\begin{figure*}[!htb]
  Structural Rules.
  \begin{center} \ccpInfAx    \ccpInfCycle \end{center}\vspace*{1\baselineskip}
  \begin{center} \ccpInfMix   \ccpInfHalt  \end{center}\vspace*{1\baselineskip}

  Logical Rules.
  \begin{center} \ccpInfTens  \ccpInfParr  \end{center}\vspace*{1\baselineskip}
  \begin{center} \ccpInfOne   \ccpInfBot   \end{center}\vspace*{1\baselineskip}
  \begin{center} \ccpInfPlus1 \ccpInfPlus2 \end{center}\vspace*{1\baselineskip}
  \begin{center} \ccpInfWith               \end{center}\vspace*{1\baselineskip}
  \begin{center} \ccpInfNil   \ccpInfTop   \end{center} 
  
  \caption{Priority-based Classical Processes}
  \label{fig:ccp}
\end{figure*}
%%% Local Variables:
%%% TeX-master: "main"
%%% End:


In this section, we will explore a revision of \ccp which has two benefits:
\begin{itemize}
\item
  we allow our priorities to be any partial order, instead of requiring them to
  be natural numbers;
\item
  and we generate a set of priority constraints during type checking, and return 
  this set as a result of type checking, instead of interleaving type and
  priority checking.
\end{itemize}
Why are these benefits?

The priority constraints are an encoding of (a sub-graph\footnote{%
  It is a sub-graph because some of the edges are enforced by the term syntax,
  and as such these are not checked by the priority constraints.
  For instance, in
  $\tm{\piBoundSend{x^{\cs{o}}}{y}{\piRecv{x^{\cs{o'}}}{z}{P}}}$,
  there is no reason to check that $\cs{o < o'}$, as there is no way for the
  receive action to take place before the send action.
} of) the communication graph, where $\cs{o < o'}$ means that the action
associated with $\cs{o}$ must happen before the action associated with
$\cs{o'}$.
In the case of parallel actions, we will have $\cs{o \parallel o'}$---i.e.\ %
there is \emph{no relation} between $\cs{o}$ and $\cs{o'}$. 
Natural numbers, however, are a total order, meaning that we will have to choose
either $\cs{o < o'}$, $\cs{o' < o}$, or $\cs{o = o'}$.
Therefore, by using priorities which have more structure than is required, we
lose some precision.

Gathering all priority constraints, instead of checking them during type
checking, makes it convenient to talk about
\begin{itemize}
\item 
  programs which are well-typed but (potentially) contain deadlocks;
\item
  programs which are deadlock free but also obey additional constraints on the
  ordering of actions, such as those that may arise from global types; 
\item
  translations between \ccp and other systems, such as \cp, because it separates
  the proofs of type preservation and that of constraint satisfaction.
\end{itemize}
We define two functions on priority-annotated types, in addition to the function
$\pr{\ty{A}}$ given in \cref{def:ccp-pr}:
\begin{definition}[Priority erasure]\label{def:cccp-fg}
  We define the function $\fg{\cdot}$ which erases priorities from types:
  \begin{gather*}
    \setlength{\arraycolsep}{1pt}
    \begin{array}{lclclcl}
      {}\fg{\ty{\one^{\cs{o}}}}      & = & \ty{\one} & \qquad
      & \fg{\ty{\nil^{\cs{o}}}}      & = & \ty{\nil} \\
      {}\fg{\ty{\bot^{\cs{o}}}}      & = & \ty{\bot} & \qquad
      & \fg{\ty{\top^{\cs{o}}}}      & = & \ty{\top} \\ 
      {}\fg{\ty{A \tens^{\cs{o}} B}} & = & \ty{\fg{A} \tens \fg{B}} & \qquad
      & \fg{\ty{A \plus^{\cs{o}} B}} & = & \ty{\fg{A} \plus \fg{B}} \\
      {}\fg{\ty{A \parr^{\cs{o}} B}} & = & \ty{\fg{A} \parr \fg{B}} & \qquad
      & \fg{\ty{A \with^{\cs{o}} B}} & = & \ty{\fg{A} \with \fg{B}}
    \end{array}
  \end{gather*}
\end{definition}
%%% Local Variables:
%%% TeX-master: "main"
%%% End:

\begin{definition}[Point-wise priority unification]\label{def:cccp-un}
  We define the function $\un{\cdot}{\cdot}$ which generates a constraint set to
  point-wise unify the priorities in two dual types:
  \begin{gather*}
    \setlength{\arraycolsep}{1pt}
    \begin{array}{lclclcl}
      \un{\one^{\cs{o}}}{\bot^{\cs{o'}}}
      & = & \cs{\{ o = o' \}} & \qquad
      \un{( {A \tens^{\cs{o}} B} )}{( {A' \parr^{\cs{o'}} B'} )}
      & = & \cs{\{ o = o' \} \cup (\un{A}{A'}) \cup (\un{B}{B'}) } \\
      \un{\bot^{\cs{o}}}{\one^{\cs{o'}}}
      & = & \cs{\{ o = o' \}} & \qquad
      \un{( {A \parr^{\cs{o}} B} )}{( {A' \tens^{\cs{o'}} B'} )}
      & = & \cs{\{ o = o' \} \cup (\un{A}{A'}) \cup (\un{B}{B'}) } \\
      \un{\nil^{\cs{o}}}{\bot^{\cs{o'}}}
      & = & \cs{\{ o = o' \}} & \qquad
      \un{( {A \plus^{\cs{o}} B} )}{( {A' \with^{\cs{o'}} B'} )}
      & = & \cs{\{ o = o' \} \cup (\un{A}{A'}) \cup (\un{B}{B'}) } \\
      \un{\top^{\cs{o}}}{\one^{\cs{o'}}}
      & = & \cs{\{ o = o' \}} & \qquad
      \un{( {A \with^{\cs{o}} B} )}{( {A' \plus^{\cs{o'}} B'} )}
      & = & \cs{\{ o = o' \} \cup (\un{A}{A'}) \cup (\un{B}{B'}) } 
    \end{array}
  \end{gather*}
\end{definition}
%%% Local Variables:
%%% TeX-master: "main"
%%% End:

\begin{definition}[Typing judgements]\label{def:ccp-typing}
  A typing judgement
  $\cseq[P]{\tmty{x_1}{A_1}\dots\tmty{x_n}{A_n}}{\mathcal{C}}$ denotes that the
  process $\tm{P}$ communicates along channels $\tm{x_1}\dots\tm{x_n}$ following
  protocols $\ty{A_1}\dots\ty{A_n}$, and that $\tm{P}$ is free from deadlocks if
  the constraint set $\cs{\mathcal{C}}$ is satisfiable.
  Typing judgements can be constructed using the inference rules in
  \cref{fig:ccp}, where $\cs{\cmin{o}{\Gamma}} = \cs{\{ o < \pr{\ty{A}} \mid
    \ty{A} \in \ty{\Gamma} \}}$.
\end{definition}
\begin{figure*}[!htb]
  Structural Rules.
  \begin{center} \cccpInfAx    \cccpInfCycle \end{center}\vspace*{1\baselineskip}
  \begin{center} \cccpInfMix   \cccpInfHalt  \end{center}\vspace*{1\baselineskip}
  
  Logical Rules.
  \begin{center} \cccpInfTens  \cccpInfParr  \end{center}\vspace*{1\baselineskip}
  \begin{center} \cccpInfOne   \cccpInfBot   \end{center}\vspace*{1\baselineskip}
  \begin{center} \cccpInfPlus1 \cccpInfPlus2 \end{center}\vspace*{1\baselineskip}
  \begin{center} \cccpInfWith                \end{center}\vspace*{1\baselineskip}
  \begin{center} \cccpInfNil   \cccpInfTop   \end{center} 

  \caption{Priority-based Classical Processes Revisited}
  \label{fig:cccp}
\end{figure*}
%%% Local Variables:
%%% TeX-master: "main"
%%% End:


\section{Safe Access Points}
\begin{definition}[Terms]\label{def:safe-ap-terms}
  We extend the terms from \cref{def:hccp-terms} with the following construct:
  \begin{gather*}
    \begin{aligned}
      \tm{P}, \tm{Q}, \tm{R}
          :=& \; \dots
      \\\mid& \; \tm{\apSend{x}{y}{P}} &&\text{connect to access point}
      \\\mid& \; \tm{\apRecv{x}{y}{P}} &&\text{connect to access point, dual}
    \end{aligned}
  \end{gather*}
\end{definition}
%%% Local Variables:
%%% TeX-master: "main"
%%% End:

\begin{definition}[Structural congruence]\label{def:safeap-equiv}
  We define the structural congruence $\equiv$ as the congruence closure over
  terms which satisfies the axioms in \cref{def:ccp-equiv} and the following
  additional axioms:
  \begin{gather*}
    \begin{array}{llcll}
      \apEquivGCNew
      & \tm{\piNew{xy}{P}}
      & \equiv
      & \tm{P}
      & \text{if }\tm{x},\tm{y}\not\in\tm{P}
    \end{array}
  \end{gather*}
\end{definition}
%%% Local Variables:
%%% TeX-master: "main"
%%% End:

\begin{definition}[Reduction]\label{def:safeap-reduction}
  A reduction $\reducesto{P}{Q}$ denotes that the process $\tm{P}$ can reduce
  to the process $\tm{Q}$ in a single step. Reductions can only be constructed
  using the rules in \cref{def:cccp-reduction} and the following addition
  congruence rule:
  \begin{gather*}
    \begin{array}{llcll}
      \apRedSessInit
      & \tm{\piNew{xy}{(\piPar{\piPar{\apSend{x}{z}{P}}{\apRecv{y}{w}{Q}}}{R})}}
      & \Longrightarrow
      & \tm{\piNew{xy}{\piNew{zw}{(\piPar{\piPar{P}{Q}}{R})}}}
      \\
    \end{array}
  \end{gather*}
  The relations $\Longrightarrow^{?}$, $\Longrightarrow^{+}$, and
  $\Longrightarrow^\star$ are the reflexive, the transitive, and the reflexive,
  transitive closures of $\Longrightarrow$.
\end{definition}
%%% Local Variables:
%%% TeX-master: "main"
%%% End:

\begin{definition}[Types]\label{def:safeap-types}
  \begin{gather*}
    \begin{aligned}
      \ty{A}, \ty{B}, \ty{C}
           :=& \; \dots
      \\ \mid& \; \ty{\apBang{n}A} &&\text{access point used by $n$ processes}
      \\ \mid& \; \ty{\apYnot{n}A} &&\text{access point used by $n$ processes, dual}
    \end{aligned}
  \end{gather*}
\end{definition}
%%% Local Variables:
%%% TeX-master: "main"
%%% End:

\begin{definition}[Typing judgements]\label{def:ap-typing}
  A typing judgement
  $\cseq[P]{\tmty{x_1}{A_1}\dots\tmty{x_n}{A_n}}{\mathcal{C}}$ denotes that the
  process $\tm{P}$ communicates along channels $\tm{x_1}\dots\tm{x_n}$ following
  protocols $\ty{A_1}\dots\ty{A_n}$, and that $\tm{P}$ is free from deadlocks if
  the constraint set $\cs{\mathcal{C}}$ is satisfiable.
  Typing judgements can be constructed using the inference rules in
  \cref{fig:cccp} and those in \cref{fig:safe-ap}.
\end{definition}
\begin{figure*}[!htb]
  Logical Rules.
  \begin{center} \apInfApBang  \apInfApYnot  \end{center}\vspace*{1\baselineskip}
  \begin{center} \apInfApBangC               \end{center}\vspace*{1\baselineskip}
  \begin{center} \apInfApYnotC               \end{center}\vspace*{1\baselineskip}
  \begin{center} \apInfApBangW \apInfApYnotW \end{center}
  
  \caption{Safe Access Points}
  \label{fig:safe-ap}
\end{figure*}
%%% Local Variables:
%%% TeX-master: "main"
%%% End:


\section{Global Types for Binary Sessions}\label{sec:globaltypes}
The work by \textcite{carbone2016} introduced globally-governed \cp (\gcp).
This work introduces a multiparty extension of \cp.
Processes in \gcp are ``typed'' by global types.
This is a bit of a misnomer, as global types are a whole host of things, but
types is not one of them, and it is probably best to see global types as the
arbiter processes to which they are compiled, together with some proof of
coherence.
With the addition of global types, \gcp adds several new---and in my eyes
orthogonal---features to \cp.

\begin{itemize}
\item
  \textbf{Global types stratify the global communication order}\\
  What do we mean by this? If there are two independent communications which can
  happen, say we have
  $\tm{({\piBoundSend{x}{y}{P}}\ppar{\piBoundSend{z}{w}{Q}}\ppar\dots)}$ and
  both $\tm{\piBoundSend{x}{y}{}}$ and $\tm{\piBoundSend{z}{w}{}}$ can happen,
  then we can use global types to stratify these communications. How? The
  potential for stratification in global types arises from the fact that we are
  basically writing an arbiter process. We are writing down who sends what to
  who when. For instance, for the example above, we could imagine some global
  type written as $\tm{{x}\to{\co{x}}.{z}\to{\co{z}}.\dots}$, where
  $\tm{\co{x}}$ and $\tm{\co{z}}$ are the endpoints on the other side of
  $\tm{x}$ and $\tm{z}$. This would require that the communication on $\tm{x}$
  happens before the communication on $\tm{z}$.
\item
  \textbf{Global types allow for broadcasting}\\
  There are two kinds of broadcasting in \gcp. First, the rules for $\tens\parr$
  and $\one\bot$ give the language some sort of ``gather'' operation and a
  wait-for-all operation. However, it is a well-known fact that the the
  constructs associated with the multiplicative connectives only perform
  ``plumbing''. Therefore, the gather construct that is implemented is not
  really gathering any information. Rather, it allows for some measure of
  abstraction in putting together sessions in which one process communicates
  with multiple other processes.

  The \emph{real} broadcasting operation is implemented by the $\plus\with$
  rule, which facilitates a single process sending one bit of information,
  identical in each case, to multiple recipients.

  The coherence rules presented by \textcite{carbone2016} are all asymmetric. In
  each case, \emph{one} process on the one side is communicating with multiple
  processes on the other. In each case, the side with \emph{one} process happens
  to be the ``strong'' side---the side which decides in which order
  communications happen $(\parr)$, and the side which makes the choice
  $(\plus)$. This, of course, is no coincidence. Instead, it has a much deeper
  connection to the logic. In each case, these stronger connectives are
  associated with \emph{unary} instead of binary rules. This makes it easy to
  expand a ``gather'' into a series of receive actions, and a choice into a
  series of choices.\footnote{This means that it should be possible to define a
    globally-governed variant of, say, \hccp or \ccp, in which both the plumbing
    and the choice are taken care of by broadcasting operations.}
\item
  \textbf{Global types allow for \emph{some} measure of non-determinism}\\
  If we see a broadcast operation as a series of binary sending operations, then
  we wouldn't expect those to be ordered among one another. The semantics for
  \gcp treat broadcast and gather as a \emph{single} operation, which only takes
  place when all intended recipients are ready. The semantics for \gcp, when
  compiled to \cp, however, imposes an ordering on the broadcasting (or
  gathering) process, by compiling the single action down to a series of
  actions. This is a bit odd.
\end{itemize}
We can easily model stratification of the global communication order in \cccp,
by simply making more demands of the final constraint set.
For instance, if we have a global type which requires
$\tm{{x}\to{\co{x}}.{z}\to{\co{z}}.\dots}$, i.e. that the first action on
$\tm{x}$ happens before the first action on $\tm{z}$, we can simply grab the
priorities associated with those actions---say $\cs{o}$ and $\cs{o'}$---and
require that the constraint $\cs{o} < \cs{o'}$ is present in the final
constraint set.

The broadcasting behaviours added by \gcp are\dots


\begin{definition}[Global types]\label{def:globaltypes}
  We define global types following \textcite{carbone2016}:
  \begin{gather*}
    \begin{array}{lcll}
      \tm{G}, \tm{H}
      & :=   & \gtyAx{x}{A}{y}                                         & \text{link} \\
      & \mid & \gtyTensParr{x}{y}{G}{H}                                & \text{multiparty gather} \\
      & \mid & \gtyOneBot{x}{y}                                        & \text{multiparty halt} \\
      & \mid & \gtyPlusWith{x}{y}{G}{H}                                & \text{multiparty choice} \\
      & \mid & \gtyNilTop{x}{y}                                        & \text{multiparty absurd}
    \end{array}
  \end{gather*}
\end{definition}

\begin{definition}[Coherence]\label{def:globaltypes-coherence}
  We define coherence following \textcite{carbone2016}:
  \begin{center} \gcpInfAx                     \end{center}\vspace{1\baselineskip}
  \begin{center} \gcpInfTensParr \gcpInfOneBot \end{center}
\end{definition}


\printbibliography

\end{document}
%%% Local Variables:
%%% TeX-master: "main"
%%% End:

\begin{lemma}\label{lem:hcp-preservation-equiv}
  If $\tm{P}\equiv\tm{Q}$, then $\seq[P]{\mathcal{G}}$ iff $\seq[Q]{\mathcal{G}}$.
\end{lemma}
\begin{proof}
  By induction on the structure of $\tm{P}\equiv\tm{Q}$.
  \begin{itemize}
  \item
    Case \hcpEquivLinkComm, \hcpEquivCutComm, and \hcpEquivCutAss1.
    As \cref{lem:cp-preservation-equiv}.
  \item
    Case \hcpEquivMixComm.
    We have to show $\seq[\piPar{P}{Q}]{\mathcal{G}} \Rightarrow
    \seq[\piPar{Q}{P}]{\mathcal{G}}$.
    For each direction, we continue by inversion on the typing derivation of the
    premise.
    The only typing derivation ends in \textsc{H-Mix}.
    We rewrite as follows:
    \[
      \begin{array}{lcl}
        \AXC{$\seq[P]{ \mathcal{G}_1 }$}
        \AXC{$\seq[Q]{ \mathcal{G}_2 }$}
        \NOM{H-Mix}
        \BIC{$\seq[\piPar{P}{Q}]{ \mathcal{G}_1 \hsep \mathcal{G}_2 }$}
        \DisplayProof
        & \Leftrightarrow
        & \AXC{$\seq[Q]{ \mathcal{G}_2 }$}
          \AXC{$\seq[P]{ \mathcal{G}_1 }$}
          \NOM{H-Mix}
          \BIC{$\seq[\piPar{Q}{P}]{ \mathcal{G}_1 \hsep \mathcal{G}_2 }$}
          \DisplayProof
      \end{array}
    \]
  \item
    Case \hcpEquivMixAss1.
    We have to show $\seq[\piPar{(\piPar{P}{Q})}{R}]{\mathcal{G}} \Rightarrow
    \seq[\piPar{(\piPar{P}{Q})}{R}]{\mathcal{G}}$.
    For each direction, we continue by inversion on the typing derivation of the
    premise.
    The only typing derivation ends in two applications of \textsc{H-Mix}.
    We rewrite as follows:
    \[
      \begin{array}{c}
        \AXC{$\seq[P]{ \mathcal{G}_1 }$}
        \AXC{$\seq[Q]{ \mathcal{G}_2 }$}
        \NOM{H-Mix}
        \BIC{$\seq[\piPar{P}{Q}]{ \mathcal{G}_1 \hsep \mathcal{G}_2 }$}
        \AXC{$\seq[R]{ \mathcal{G}_3 }$}
        \NOM{H-Mix}
        \BIC{$\seq[\piPar{(\piPar{P}{Q})}{R}]{ \mathcal{G}_1 \hsep
        \mathcal{G}_2 \hsep \mathcal{G}_3 }$}
        \DisplayProof
        \\\\
        \Updownarrow
        \\\\
        \AXC{$\seq[P]{ \mathcal{G}_1 }$}
        \AXC{$\seq[Q]{ \mathcal{G}_2 }$}
        \AXC{$\seq[R]{ \mathcal{G}_3 }$}
        \NOM{H-Mix}
        \BIC{$\seq[\piPar{Q}{R}]{ \mathcal{G}_2 \hsep \mathcal{G}_3 }$}
        \NOM{H-Mix}
        \BIC{$\seq[\piPar{(\piPar{P}{Q})}{R}]{ \mathcal{G}_1 \hsep
        \mathcal{G}_2 \hsep \mathcal{G}_3 }$}
        \DisplayProof
      \end{array}
    \]
  \item
    Case \hcpEquivMixCut1.
    We have to show $\seq[\piPar{P}{\cpCut{x}{Q}{R}}]{\mathcal{G}} \Rightarrow
    \seq[\piPar{P}{\cpCut{x}{Q}{R}}]{\mathcal{G}}$.
    For each direction, we continue by inversion on the typing derivation of the
    premise.
    The only typing derivation ends in \textsc{H-Mix} and \textsc{Cut}.
    We rewrite as follows:
    \[
      \begin{array}{c}
        \AXC{$\seq[P]{ \mathcal{G}_1 }$}
        \AXC{$\seq[Q]{ \mathcal{G}_2 \hsep \Gamma, \tmty{x}{A} }$}
        \AXC{$\seq[R]{ \mathcal{G}_3 \hsep \Delta, \tmty{x}{A^\bot} }$}
        \NOM{Cut}
        \BIC{$\seq[\cpCut{x}{Q}{R}]{ \mathcal{G}_2 \hsep \mathcal{G}_3 \hsep
        \Gamma, \Delta }$}
        \NOM{H-Mix}
        \BIC{$\seq[\piPar{P}{\cpCut{x}{Q}{R}}]{ \mathcal{G}_1 \hsep
        \mathcal{G}_2 \hsep \mathcal{G}_3 \hsep \Gamma, \Delta }$}
        \DisplayProof
        \\\\
        \Updownarrow
        \\\\
        \AXC{$\seq[P]{ \mathcal{G}_1 }$}
        \AXC{$\seq[Q]{ \mathcal{G}_2 \hsep \Gamma, \tmty{x}{A} }$}
        \NOM{H-Mix}
        \BIC{$\seq[\piPar{P}{\cpCut{x}{Q}{R}}]{ \mathcal{G}_1 \hsep
        \mathcal{G}_2 \hsep \Gamma, \tmty{x}{A} }$}
        \AXC{$\seq[R]{ \mathcal{G}_3 \hsep \Delta, \tmty{x}{A^\bot} }$}
        \NOM{Cut}
        \BIC{$\seq[\piPar{P}{\cpCut{x}{Q}{R}}]{ \mathcal{G}_1 \hsep
        \mathcal{G}_2 \hsep \mathcal{G}_3 \hsep \Gamma, \Delta }$}
        \DisplayProof
      \end{array}
    \]
  \item
    Case \hcpEquivMixHalt1.
    We have to show that if $\seq[(\piPar{P}{\piHalt})]{\mathcal{G}}$, then
    $\seq[P]{\mathcal{G}}$, and the converse.
    In the $[\Rightarrow]$ direction, we proceed by inversion on the typing
    derivation of the premise.
    The only derivation ends in \textsc{H-Mix} and \textsc{H-Halt}.
    In the $[\Leftarrow]$ direction, we rewrite immediately.
    For either direction, we rewrite as follows:
    \[
      \begin{array}{lcl}
        \AXC{$\seq[P]{\mathcal{G}}$}
        \AXC{}
        \NOM{H-Halt}
        \UIC{$\seq[\piHalt]{\emptyhypercontext}$}
        \NOM{H-Mix}
        \BIC{$\seq[(\piPar{P}{\piHalt})]{\mathcal{G}}$}
        \DisplayProof
        & \Leftrightarrow
        & \AXC{$\seq[P]{\mathcal{G}}$}
          \DisplayProof
      \end{array}
    \]
  \end{itemize}
\end{proof}
%%% Local Variables:
%%% TeX-master: "main"
%%% End:


\begin{lemma}\label{lem:hccp-preservation-equiv}
  If $\tm{P}\equiv\tm{Q}$, then $\seq[P]{\mathcal{G}}$ iff $\seq[Q]{\mathcal{G}}$.
\end{lemma}
\begin{proof}
  By induction on the structure of $\tm{P}\equiv\tm{Q}$.
  \begin{itemize}
  \item
    Case \cpEquivLinkComm.
    We have to show $\seq[\cpLink{x}{y}]{\mathcal{G}} \Leftrightarrow
    \seq[\cpLink{y}{x}]{\mathcal{G}}$.
    For each direction, we continue by inversion on the typing derivation of the premise.
    The only typing derivation ends in \textsc{Ax}.
    For each direction, we rewrite as follows:
    \[
      \begin{array}{lcl}
        \AXC{}
        \NOM{Ax}
        \UIC{$\seq[\cpLink{x}{y}]{\tmty{x}{A}, \tmty{y}{A^\bot}}$}
        \DisplayProof
        & \Longrightarrow
        & \AXC{}
          \NOM{Ax}
          \UIC{$\seq[\cpLink{y}{x}]{\tmty{y}{A^\bot}, \tmty{x}{A^{\bot\bot}}}$}
          \NOM{\cref{lem:cp-negation-involutive}}
          \UIC{$\seq[\cpLink{y}{x}]{\tmty{y}{A^\bot}, \tmty{x}{A}}$}
          \DisplayProof
      \end{array}
    \]
  \item
    Case \hccpEquivMixComm.
    We have to show $\seq[(\piPar{P}{Q})]{\mathcal{G}} \Leftrightarrow
    \seq[(\piPar{Q}{P})]{\mathcal{G}}$.
    For each direction, we continue by inversion on the typing derivation of the
    premise.
    The only typing derivation ends in \textsc{H-Mix}.
    We rewrite as follows:
    \[
      \begin{array}{lcl}
        \AXC{$\seq[P]{\mathcal{G}}$}
        \AXC{$\seq[Q]{\mathcal{H}}$}
        \NOM{H-Mix}
        \BIC{$\seq[(\piPar{P}{Q})]{\mathcal{G} \hsep \mathcal{H}}$}
        \DisplayProof
        & \Leftrightarrow
        & \AXC{$\seq[Q]{\mathcal{H}}$}
          \AXC{$\seq[P]{\mathcal{G}}$}
          \NOM{H-Mix}
          \BIC{$\seq[(\piPar{Q}{P})]{\mathcal{H} \hsep \mathcal{G}}$}
          \DisplayProof
      \end{array}
    \]
  \item
    Case \hccpEquivMixAss1.
    We have to show $\seq[(\piPar{P}{(\piPar{Q}{R})})]{\mathcal{G}}
    \Leftrightarrow \seq[(\piPar{(\piPar{P}{Q})}{R})]{\mathcal{G}}$.
    For each direction, we continue by inversion on the typing derivation of the
    premise. 
    The only typing derivation ends in two applications of \textsc{H-Mix}.
    We rewrite as follows:
    \[
      \begin{array}{c}
        \AXC{$\seq[P]{\mathcal{G}_1}$}
        \AXC{$\seq[Q]{\mathcal{G}_2}$}
        \AXC{$\seq[R]{\mathcal{G}_3}$}
        \NOM{H-Mix}
        \BIC{$\seq[(\piPar{Q}{R})]{\mathcal{G}_2\hsep\mathcal{G}_3}$}
        \NOM{H-Mix}
        \BIC{$\seq[(\piPar{P}{(\piPar{Q}{R})})]{\mathcal{G}_1\hsep\mathcal{G}_2\hsep\mathcal{G}_3}$}
        \DisplayProof
        \\\\
        \Updownarrow
        \\\\
        \AXC{$\seq[P]{\mathcal{G}_1}$}
        \AXC{$\seq[Q]{\mathcal{G}_2}$}
        \NOM{H-Mix}
        \BIC{$\seq[(\piPar{P}{Q})]{\mathcal{G}_1\hsep\mathcal{G}_2}$}
        \AXC{$\seq[R]{\mathcal{G}_3}$}
        \NOM{H-Mix}
        \BIC{$\seq[(\piPar{P}{(\piPar{Q}{R})})]{\mathcal{G}_1\hsep\mathcal{G}_2\hsep\mathcal{G}_3}$}
        \DisplayProof
      \end{array}
    \]
  \item
    Case \hccpEquivNewComm.
    We have to show $\seq[\piNew{x}{\piNew{y}{P}}]{\mathcal{G}} \Leftrightarrow
    \seq[\piNew{y}{\piNew{x}{P}}]{\mathcal{G}}$.
    For each direction, we continue by inversion on the typing derivation of the premise.
    There are two typing derivations, each ends in two applications of \textsc{H-Cycle}.
    \begin{itemize}
    \item
      If $\tm{x}$ and $\tm{y}$ occur together in the same environment, we
      rewrite as follows:
      \[
        \begin{array}{c}
          \AXC{$\seq[P]{\mathcal{G} \hsep \tmty{x}{A}, \Gamma \hsep
          \tmty{x}{A^\bot}, \tmty{y}{B}, \Delta \hsep \tmty{y}{B^\bot},
          \Theta}$} 
          \NOM{H-Cycle}
          \UIC{$\seq[\piNew{y}{P}]{\mathcal{G} \hsep \tmty{x}{A}, \Gamma \hsep
          \tmty{x}{A^\bot}, \Delta, \Theta}$}
          \NOM{H-Cycle}
          \UIC{$\seq[\piNew{x}{\piNew{y}{P}}]{\mathcal{G} \hsep \Gamma, \Delta,
          \Theta}$} 
          \DisplayProof
          \\\\
          \Updownarrow
          \\\\
          \AXC{$\seq[P]{\mathcal{G} \hsep \tmty{x}{A}, \Gamma \hsep
          \tmty{x}{A^\bot}, \tmty{y}{B}, \Delta \hsep \tmty{y}{B^\bot},
          \Theta}$}
          \NOM{H-Cycle}
          \UIC{$\seq[\piNew{x}{P}]{\mathcal{G} \hsep \Gamma, \tmty{y}{B}, \Delta
          \hsep \tmty{y}{B^\bot}, \Theta}$}
          \NOM{H-Cycle}
          \UIC{$\seq[\piNew{x}{\piNew{y}{P}}]{\mathcal{G} \hsep \Gamma, \Delta,
          \Theta}$}
          \DisplayProof
        \end{array}
      \]
    \item
      Otherwise, we rewrite as follows:
      \[
        \begin{array}{c}
          \AXC{$\seq[P]{\mathcal{G} \hsep \tmty{x}{A}, \Gamma_1 \hsep
          \tmty{x}{A^\bot}, \Gamma_2 \hsep \tmty{y}{B}, \Delta_1 \hsep
          \tmty{y}{B^\bot}, \Delta_2}$}
          \NOM{H-Cycle}
          \UIC{$\seq[\piNew{y}{P}]{\mathcal{G} \hsep \tmty{x}{A}, \Gamma_1 \hsep
          \tmty{x}{A^\bot}, \Gamma_2 \hsep \Delta_1, \Delta_2}$} 
          \NOM{H-Cycle}
          \UIC{$\seq[\piNew{x}{\piNew{y}{P}}]{\mathcal{G} \hsep \Gamma_1,
          \Gamma_2 \hsep \Delta_1, \Delta_2}$}
          \DisplayProof
          \\\\
          \Updownarrow
          \\\\
          \AXC{$\seq[P]{\mathcal{G} \hsep \tmty{x}{A}, \Gamma_1 \hsep
          \tmty{x}{A^\bot}, \Gamma_2 \hsep \tmty{y}{B}, \Delta_1 \hsep
          \tmty{y}{B^\bot}, \Delta_2}$}
          \NOM{H-Cycle}
          \UIC{$\seq[\piNew{x}{P}]{\mathcal{G} \hsep \Gamma_1, \Gamma_2 \hsep
          \tmty{y}{B}, \Delta_1 \hsep \tmty{y}{B^\bot}, \Delta_2}$} 
          \NOM{H-Cycle}
          \UIC{$\seq[\piNew{x}{\piNew{y}{P}}]{\mathcal{G} \hsep \Gamma_1,
          \Gamma_2 \hsep \Delta_1, \Delta_2}$}
          \DisplayProof
        \end{array}
      \]
    \end{itemize}
  \item
    Case \hccpEquivScopeExt1.
    We have that $\notFreeIn{x}{P}$.
    We have to show $\seq[\piNew{x}{(\piPar{P}{Q})}]{\mathcal{G}}
    \Leftrightarrow \seq[(\piPar{P}{\piNew{x}{Q}})]{\mathcal{G}}$.
    For each direction, we continue by inversion on the typing derivation of the
    premise. 
    The only derivation ends in \textsc{H-Cycle} and \textsc{H-Mix}.
    We rewrite as follows:
    \[
      \begin{array}{c}
        \AXC{$\seq[P]{\mathcal{G}}$}
        \AXC{$\seq[Q]{\mathcal{H} \hsep \tmty{x}{A}, \Gamma \hsep
        \tmty{x}{A^\bot}, \Delta}$}
        \NOM{H-Mix}
        \BIC{$\seq[(\piPar{P}{Q})]{\mathcal{G} \hsep \mathcal{H} \hsep
        \tmty{x}{A}, \Gamma \hsep \tmty{x}{A^\bot}, \Delta}$} 
        \NOM{H-Cycle}
        \UIC{$\seq[\piNew{x}{(\piPar{P}{Q})}]{\mathcal{G} \hsep \mathcal{H}
        \hsep \Gamma, \Delta}$}
        \DisplayProof
        \\\\
        \Updownarrow
        \\\\
        \AXC{$\seq[P]{\mathcal{G}}$}
        \AXC{$\seq[Q]{\mathcal{H} \hsep \tmty{x}{A}, \Gamma \hsep
        \tmty{x}{A^\bot}, \Delta}$}
        \NOM{H-Cycle}
        \UIC{$\seq[\piNew{x}{Q}]{\mathcal{H} \hsep \tmty{x}{A}, \Gamma \hsep
        \tmty{x}{A^\bot}, \Delta}$}
        \NOM{H-Mix}
        \BIC{$\seq[(\piPar{P}{\piNew{x}{Q}})]{\mathcal{G} \hsep \mathcal{H}
        \hsep \Gamma, \Delta}$}
        \DisplayProof
      \end{array}
    \]
  \item
    Case \hccpEquivMixHalt1.
    We have to show that if $\seq[(\piPar{P}{\piHalt})]{\mathcal{G}}$, then
    $\seq[P]{\mathcal{G}}$, and the converse.
    In the $[\Rightarrow]$ direction, we proceed by inversion on the typing
    derivation of the premise.
    The only derivation ends in \textsc{H-Mix} and \textsc{H-Halt}.
    In the $[\Leftarrow]$ direction, we rewrite immediately.
    For either direction, we rewrite as follows:
    \[
      \begin{array}{lcl}
        \AXC{$\seq[P]{\mathcal{G}}$}
        \AXC{}
        \NOM{H-Halt}
        \UIC{$\seq[\piHalt]{\emptyhypercontext}$}
        \NOM{H-Mix}
        \BIC{$\seq[(\piPar{P}{\piHalt})]{\mathcal{G}}$}
        \DisplayProof
        & \Leftrightarrow
        & \AXC{$\seq[P]{\mathcal{G}}$}
          \DisplayProof
      \end{array}
    \]
  \end{itemize}
  The cases for reflexivity, transitivity, symmetry, and congruence are trivial.
\end{proof}
%%% Local Variables:
%%% TeX-master: "main"
%%% End:

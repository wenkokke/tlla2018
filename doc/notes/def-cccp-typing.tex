\begin{definition}[Typing judgements]\label{def:ccp-typing}
  A typing judgement
  $\cseq[P]{\tmty{x_1}{A_1}\dots\tmty{x_n}{A_n}}{\mathcal{C}}$ denotes that the
  process $\tm{P}$ communicates along channels $\tm{x_1}\dots\tm{x_n}$ following
  protocols $\ty{A_1}\dots\ty{A_n}$, and that $\tm{P}$ is free from deadlocks if
  the constraint set $\cs{\mathcal{C}}$ is satisfiable.
  Typing judgements can be constructed using the inference rules in
  \cref{fig:ccp}, where $\cs{\cmin{o}{\Gamma}} = \cs{\{ o < \pr{\ty{A}} \mid
    \ty{A} \in \ty{\Gamma} \}}$.
\end{definition}
\begin{figure*}[!htb]
  Structural Rules.
  \begin{center} \cccpInfAx    \cccpInfCycle \end{center}\vspace*{1\baselineskip}
  \begin{center} \cccpInfMix   \cccpInfHalt  \end{center}\vspace*{1\baselineskip}
  
  Logical Rules.
  \begin{center} \cccpInfTens  \cccpInfParr  \end{center}\vspace*{1\baselineskip}
  \begin{center} \cccpInfOne   \cccpInfBot   \end{center}\vspace*{1\baselineskip}
  \begin{center} \cccpInfPlus1 \cccpInfPlus2 \end{center}\vspace*{1\baselineskip}
  \begin{center} \cccpInfWith                \end{center}\vspace*{1\baselineskip}
  \begin{center} \cccpInfNil   \cccpInfTop   \end{center} 

  \caption{Priority-based Classical Processes Revisited}
  \label{fig:cccp}
\end{figure*}
%%% Local Variables:
%%% TeX-master: "main"
%%% End:

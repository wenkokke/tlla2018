\begin{theorem}[Progress]\label{thm:hccp-progress}
  If $\seq[P]{\mathcal{G}}$, then there exists a $\tm{Q}$ such that either
  $\tm{P}\equiv\tm{Q}$ and $\tm{Q}$ is in canonical form, or $\reducesto{P}{Q}$.
\end{theorem}
\begin{proof}
  By induction on the structure of the derivation for $\seq[P]{\mathcal{G}}$.
  The only interesting cases are when the last rule of the derivation is
  \textsc{H-Cycle} or \textsc{H-Mix}. In every other case, the typing rule
  constructs a term which is in canonical form. 

  If the last rule in the derivation is \textsc{H-Cycle} or \textsc{H-Mix}, we
  consider the maximum evaluation prefix $\tm{G}$ of $\tm{P}$.
  Let the prefix $\tm{G}$ consist of $n$ cycles, and $m$ mixes, which introduces
  $n$ channels, and composes $m+1$ actions.
  Furthermore, let $o$ denote the number of environments in the
  hyper-environment $\ty{\mathcal{G}}$.

  Actions are introduced by the logical rules, each of which is annotated with
  the side condition that to act on one end-point of a channel, the other
  end-point may not be in scope.
  Let us call a channel ``locked'' if both its end-points are in scope.
  In this sense, a mix can unlock \emph{one} channel introduced by a cycle,
  and $m$ mixes can unlock $m$ channels.
  As the hyper-environment $\ty{\mathcal{G}}$ consists of $o$ environments,
  there are $o - 1$ possible applications of \textsc{Mix} which do not unlock
  one of the $n$ channels introduced in $\tm{G}$.
  Let $n'$ refer to the number of unlocked channels.
  We have $m - (o - 1) \le n' \le m$.
  Therefore, one of the following must be true:
  \begin{itemize}
  \item
    One of the actions composed by $\tm{G}$ is a link $\tm{\cpLink{x}{y}}$
    acting on a bound channel.
    Suppose that $\tm{x}$ is the bound channel.
    There exist evaluation contexts $\tm{E}$, $\tm{E'}$, and $\tm{E''}$ such that
    \[
      \tm{P} =
      \tm{\cpPlug{E}{\piNew{x}{\cpPlug{E'}{\piPar{\cpPlug{E''}{\cpLink{x}{y}}}{R}}}}}.
    \]
    We rewrite by $\equiv$ to obtain
    \[
      \tm{P} \equiv
      \tm{\cpPlug{E}{\cpPlug{E'}{\cpPlug{E''}{\piNew{x}{( \piPar{\cpLink{x}{y}}{R} )}}}}}.
    \]
    We then reduce by applying \hccpRedAxCut1.
    Similarly if $\tm{y}$ is the bound channel.
  \item
    Two of the actions composed by $\tm{G}$ are acting on the \emph{same}
    bound channel.
    Let the two actions be $\tm{P_i}$ and $\tm{P_i}$, and the bound channel
    $\tm{x}$.
    There exist evaluation contexts $\tm{E}$, $\tm{E'}$, $\tm{E_i}$, and
    $\tm{E_j}$ such that
    \[
      \tm{P} =
      \tm{\cpPlug{E}{
          \piNew{x}{\cpPlug{E'}{
              \piPar{
                \cpPlug{E_i}{P_i}
              }{
                \cpPlug{E_j}{P_j}
              }}}}}
    \]
    We rewrite by $\equiv$ to obtain
    \[
      \tm{P} \equiv
      \tm{\cpPlug{E}{
          \cpPlug{E'}{
            \cpPlug{E_i}{
              \cpPlug{E_j}{
                \piNew{x}{(\piPar{P_i}{P_j})}
              }}}}}
    \]
    We then reduce by applying the appropriate \textbeta-reduction.
  \item
    Otherwise, (at least) one of the actions composed by $\tm{G}$ must be acting
    on a free channel.
    None of the actions composed by $\tm{G}$ is a link acting on a bound channel.
    No two actions composed by $\tm{G}$ are acting on the same bound channel.
    Therefore, $\tm{P}$ is equivalent to a process in canonical form.
  \end{itemize}
\end{proof}
%%% Local Variables:
%%% TeX-master: "main"
%%% End:

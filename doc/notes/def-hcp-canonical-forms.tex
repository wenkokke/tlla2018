\begin{definition}[Canonical forms]\label{def:hcp-canonical-forms}
  A process $\tm{P}$ is in canonical form if it is an action, or if it is of the
  form
  \[
    \tm{(\piPar{
        \piNew{x_1}{(\piPar{P_1}{\dots \piNew{x_n}{(\piPar{P_n}{P_{n+1}})} \dots})}
      }{
        Q_1 \ppar \dots \ppar Q_m
      })},
  \]
  where each $\tm{P_i}$ and each $\tm{Q_i}$ is an action, no $\tm{P_i}$ is a
  link acting on a bound channel, and no two actions $\tm{P_i}$ and $\tm{P_j}$
  act on the same bound channel.

  An immediate consequence of this definition is that if a process is in
  canonical form, then at least one of the actions $\tm{P_i}$ and all of the
  actions $\tm{Q_i}$ are acting on free channels.
\end{definition}
%%% Local Variables:
%%% TeX-master: "main"
%%% End:

\documentclass[submission,copyright,creativecommons]{eptcs}
\providecommand{\event}{Linearity and TLLA 2018}
\usepackage{breakurl} % Not needed if you use pdflatex only.
\usepackage{underscore} % Only needed if you use pdflatex.
\usepackage{textcomp,textgreek,upgreek}
\usepackage{amsthm}
\newtheorem{definition}{Definition}
\newtheorem{lemma}{Lemma}
\newtheorem{theorem}{Theorem}
\newtheorem{corollary}{Corollary}
\input{preamble-typing}

\title{Taking Linear Logic Apart}
\author{%
  Wen Kokke
  \institute{University of Edinburgh\\ Edinburgh, Scotland}
  \email{wen.kokke@ed.ac.uk}}
\def\titlerunning{Taking Linear Logic Apart}
\def\authorrunning{Wen Kokke}

\begin{document}
\maketitle

\begin{abstract}
\end{abstract}

%% Classical Processes

\begin{definition}[Terms]\label{def:cp-terms}
  \begin{gather*}
    \begin{array}{rllrll}
      \tm{P}, \tm{Q}, \tm{R}
           :=& \tm{\cpLink{x}{y}}       &\text{link}
      &  \mid& \tm{\cpCut{x}{P}{Q}}     &\text{parallel composition, ``cut''}
      \\ \mid& \tm{\cpSend{x}{y}{P}{Q}} &\text{``output''}
      &  \mid& \tm{\cpRecv{x}{y}{P}}    &\text{``input''}
      \\ \mid& \tm{\cpHalt{x}}          &\text{halt}
      &  \mid& \tm{\cpWait{x}{P}}       &\text{wait}
      \\ \mid& \tm{\cpInl{x}{P}}        &\text{select left choice}
      &  \mid& \tm{\cpInr{x}{P}}        &\text{select right choice}
      \\ \mid& \tm{\cpCase{x}{P}{Q}}    &\text{offer binary choice}
      &  \mid& \tm{\cpAbsurd{x}}        &\text{offer nullary choice}
    \end{array}
  \end{gather*}
\end{definition}

\begin{definition}[Structural congruence]\label{def:cp-equiv}
  We define the structural congruence $\equiv$ as the congruence closure over
  terms which satisfies the following additional axioms:
  \begin{gather*}
    \setlength{\arraycolsep}{3pt}
    \begin{array}{llcll}
      \cpEquivLinkComm
      & \tm{\cpLink{x}{y}}
      & \equiv
      & \tm{\cpLink{y}{x}}
      \\
      \cpEquivCutComm
      & \tm{\cpCut{x}{P}{Q}}
      & \equiv
      & \tm{\cpCut{x}{Q}{P}}
      \\
      \cpEquivCutAss1
      & \tm{\cpCut{x}{P}{\cpCut{y}{Q}{R}}}
      & \equiv
      & \tm{\cpCut{y}{\cpCut{x}{P}{Q}}{R}}
      & \text{if }\notFreeIn{x}{R}\text{ and }\notFreeIn{y}{P}
    \end{array}
  \end{gather*}
\end{definition}

\begin{definition}[Reduction]\label{def:cp-reduction}
  A reduction $\reducesto{P}{Q}$ denotes that the process $\tm{P}$ can reduce
  to the process $\tm{Q}$ in a single step.
  The relation $\Longrightarrow^\star$ is the reflexive, transitive closure of
  $\Longrightarrow$.
  Reductions can only be constructed as follows:
  \begin{gather*}
    \begin{array}{llcll}
      \cpRedAxCut1
      & \tm{\cpCut{x}{\cpLink{w}{x}}{P}}
      & \Longrightarrow
      & \tm{\cpSub{w}{x}{P}} 
      \\
      \cpRedBetaTensParr
      & \tm{\cpCut{x}{\cpSend{x}{y}{P}{Q}}{\cpRecv{x}{z}{R}}}
      & \Longrightarrow
      & \tm{\cpCut{y}{P}{\cpCut{x}{Q}{\cpSub{y}{z}{R}}}}
      \\
      \cpRedBetaOneBot
      & \tm{\cpCut{x}{\cpHalt{x}}{\cpWait{x}{P}}}
      & \Longrightarrow
      & \tm{P}
      \\
      \cpRedBetaPlusWith1
      & \tm{\cpCut{x}{\cpInl{x}{P}}{\cpCase{x}{Q}{R}}}
      & \Longrightarrow
      & \tm{\cpCut{x}{P}{Q}}
      \\
      \cpRedBetaPlusWith2
      & \tm{\cpCut{x}{\cpInr{x}{P}}{\cpCase{x}{Q}{R}}}
      & \Longrightarrow
      & \tm{\cpCut{x}{P}{R}}
      \\
      \cpRedKappaTens1
      & \tm{\cpCut{x}{\cpSend{y}{z}{P}{Q}}{R}}
      & \Longrightarrow \;
      & \tm{\cpSend{y}{z}{\cpCut{x}{P}{R}}{Q}}
        \quad \text{if} \; \notFreeIn{x}{Q}
      \\
      \cpRedKappaTens2
      & \tm{\cpCut{x}{\cpSend{y}{z}{P}{Q}}{R}}
      & \Longrightarrow \;
      & \tm{\cpSend{y}{z}{P}{\cpCut{x}{Q}{R}}}
        \quad \text{if} \; \notFreeIn{x}{P}
      \\
      \cpRedKappaParr
      & \tm{\cpCut{x}{\cpRecv{y}{z}{P}}{R}}
      & \Longrightarrow \;
      & \tm{\cpRecv{y}{z}{\cpCut{x}{P}{R}}}
      \\
      \cpRedKappaBot
      & \tm{\cpCut{x}{\cpWait{y}{P}}{R}}
      & \Longrightarrow \;
      & \tm{\cpWait{y}{\cpCut{x}{P}{R}}}
      \\
      \cpRedKappaPlus1
      & \tm{\cpCut{x}{\cpInl{y}{P}}{R}}
      & \Longrightarrow \;
      & \tm{\cpInl{y}{\cpCut{x}{P}{R}}}
      \\
      \cpRedKappaPlus2
      &\tm{\cpCut{x}{\cpInr{y}{P}}{R}}
      & \Longrightarrow \;
      & \tm{\cpInr{y}{\cpCut{x}{P}{R}}}
      \\
      \cpRedKappaWith
      & \tm{\cpCut{x}{\cpCase{y}{P}{Q}}{R}}
      & \Longrightarrow \;
      & \tm{\cpCase{y}{\cpCut{x}{P}{R}}{\cpCut{x}{Q}{R}}}
      \\
      \cpRedKappaTop
      & \tm{\cpCut{x}{\cpAbsurd{y}}{R}}
      & \Longrightarrow \;
      & \tm{\cpAbsurd{y}}
    \end{array}
  \end{gather*}
  \begin{center}
    \begin{prooftree*}
      \AXC{$\reducesto{P}{P^\prime}$}
      \SYM{\cpRedGammaCut}
      \UIC{$\reducesto{\cpCut{x}{P}{Q}}{\cpCut{x}{P^\prime}{Q}}$}
    \end{prooftree*}
    \begin{prooftree*}
      \AXC{$\tm{P}\equiv\tm{Q}$}
      \AXC{$\reducesto{Q}{Q^\prime}$}
      \AXC{$\tm{Q^\prime}\equiv\tm{P^\prime}$}
      \SYM{\cpRedGammaEquiv}
      \TIC{$\reducesto{P}{P^\prime}$}
    \end{prooftree*}
  \end{center}
\end{definition}

\begin{definition}[Types]\label{def:cp-types}
  \begin{gather*}
    \begin{array}{rclrcl}
      \ty{A}, \ty{B}, \ty{C}
           :=& \ty{A \tens B} &\text{pair of independent processes}
      &  \mid& \ty{\one}      &\text{unit for} \; {\tens}
      \\ \mid& \ty{A \parr B} &\text{pair of interdependent processes}
      &  \mid& \ty{\bot}      &\text{unit for} \; {\parr}
      \\ \mid& \ty{A \plus B} &\text{internal choice}
      &  \mid& \ty{\nil}      &\text{unit for} \; {\plus}
      \\ \mid& \ty{A \with B} &\text{external choice}
      &  \mid& \ty{\top}      &\text{unit for} \; {\with}
    \end{array}
  \end{gather*}
\end{definition}

\begin{figure*}[!htb]
  Structural Rules.
  \begin{center} \cpInfAx     \cpInfCut      \end{center}

  Logical Rules.
  \begin{center} \cpInfTens   \cpInfParr     \end{center}
  \begin{center} \cpInfOne    \cpInfBot      \end{center}
  \begin{center} \cpInfPlus1  \cpInfPlus2    \end{center}
  \begin{center} \cpInfWith                  \end{center}
  \begin{center} \cpInfNil    \cpInfTop      \end{center}

  \caption{Classical Processes (\cp)}
  \label{fig:cp}
\end{figure*}

%% Hypersequent Classical Processes

\begin{definition}[Terms]\label{def:hcp-terms}
  \begin{gather*}
    \begin{array}{rllrll}
      \tm{P}, \tm{Q}, \tm{R}
           :=& \tm{\cpLink{x}{y}}         &\text{link}
      &  \mid& \tm{\piHalt}               &\text{terminated process}
      \\ \mid& \tm{\piNew{x}{P}}          &\text{name restriction, ``new''}
      &  \mid& \tm{( \piPar{P}{Q} )}      &\text{parallel composition, ``mix''}
      \\ \mid& \tm{\piBoundSend{x}{y}{P}} &\text{output}
      &  \mid& \tm{\piRecv{x}{y}{P}}      &\text{input}
      \\ \mid& \tm{\piBoundSend{x}{}{P}}  &\text{``halt''}
      &  \mid& \tm{\cpWait{x}{}{P}}       &\text{wait}
      \\ \mid& \tm{\cpInl{x}{P}}          &\text{select left choice}
      &  \mid& \tm{\cpInr{x}{P}}          &\text{select right choice}
      \\ \mid& \tm{\cpCase{x}{P}{Q}}      &\text{offer binary choice}
      &  \mid& \tm{\cpAbsurd{x}}          &\text{offer nullary choice}
    \end{array}
  \end{gather*}
\end{definition}

\begin{definition}[Structural congruence]\label{def:hcp-equiv}
  We define the structural congruence $\equiv$ as the congruence closure over
  terms which satisfies the following additional axioms:
  \begin{gather*}
    \setlength{\arraycolsep}{3pt}
    \begin{array}{llcllllcll}
        \cpEquivLinkComm
      & \tm{\cpLink{x}{y}}
      & \equiv
      & \tm{\cpLink{y}{x}}
      &
      &
        \hccpEquivMixHalt1
      & \tm{\piPar{P}{\piHalt}}
      & \equiv
      & \tm{P}
      &
      \\
        \hccpEquivMixComm
      & \tm{\piPar{P}{Q}}
      & \equiv
      & \tm{\piPar{Q}{P}}
      &
      &
        \hccpEquivMixAss1
      & \tm{\piPar{P}{( \piPar{Q}{R} )}}
      & \equiv
      & \tm{\piPar{( \piPar{P}{Q} )}{R}}
      &
      \\
        \hccpEquivNewComm
      & \tm{\piNew{x}{\piNew{y}{P}}}
      & \equiv
      & \tm{\piNew{y}{\piNew{x}{P}}}
      &
      &
        \hccpEquivScopeExt1
      & \tm{\piNew{x}{( \piPar{P}{Q} )}}
      & \equiv
      & \tm{\piPar{P}{\piNew{x}{Q}}}
      & \text{if }\notFreeIn{x}{P} 
    \end{array}
  \end{gather*}
\end{definition}

\begin{definition}[Reduction]\label{def:hcp-reduction}
  A reduction $\reducesto{P}{Q}$ denotes that the process $\tm{P}$ can reduce
  to the process $\tm{Q}$ in a single step. Reductions can only be constructed
  as follows:
  \begin{gather*}
    \begin{array}{llcll}
      &
      & \vdots
      &
      \\
      \cpRedBetaTensParr
      & \tm{\cpCut{x}{\piSend{x}{y}{P}}{\piRecv{x}{z}{R}}}
      & \Longrightarrow
      & \tm{\piNew{x}{\piNew{y}{(\piPar{P}{\cpSub{y}{z}{R}})}}}
      \\
      \cpRedBetaOneBot
      & \tm{\cpCut{x}{\piSend{x}{}{P}}{\piRecv{x}{}{P}}}
      & \Longrightarrow
      & \tm{\piPar{P}{Q}}
      \\
      &
      & \vdots
      &
    \end{array}
  \end{gather*}
  \begin{center}
    \begin{prooftree*}
      \AXC{$\reducesto{P}{P^\prime}$}
      \SYM{\hccpRedGammaNew}
      \UIC{$\reducesto{\piNew{x}{P}}{\piNew{x}{P^\prime}}$}
    \end{prooftree*}
    \begin{prooftree*}
      \AXC{$\reducesto{P}{P^\prime}$}
      \SYM{\hccpRedGammaMix}
      \UIC{$\reducesto{\piPar{P}{Q}}{\piPar{P^\prime}{Q}}$}
    \end{prooftree*}
    \begin{prooftree*}
      \AXC{$\tm{P}\equiv\tm{Q}$}
      \AXC{$\reducesto{Q}{Q^\prime}$}
      \AXC{$\tm{Q^\prime}\equiv\tm{P^\prime}$}
      \SYM{\hccpRedGammaEquiv}
      \TIC{$\reducesto{P}{P^\prime}$}
    \end{prooftree*}
  \end{center}
  The relations $\Longrightarrow^{?}$, $\Longrightarrow^{+}$, and
  $\Longrightarrow^\star$ are the reflexive, transitive and reflexive,
  transitive closures of $\Longrightarrow$. 
\end{definition}

\begin{figure*}[!htb]
  Structural Rules.
  \begin{center} \hccpInfAx        \hccpInfCycle \end{center}
  \begin{center} \hccpInfMix       \hccpInfHalt  \end{center}

  Logical Rules.

  \begin{center} \hccpInfBoundTens \hccpInfParr  \end{center}
  \begin{center} \hccpInfOne       \hccpInfBot   \end{center}
  \begin{center} \hccpInfPlus1     \hccpInfPlus2 \end{center}
  \begin{center} \hccpInfWith                    \end{center}
  \begin{center} \hccpInfNil       \hccpInfTop   \end{center}

  \centering
  (Where each logical rule has the side condition that $\tm{x} \not\in
  \ty{\mathcal{G}}$.)

  \caption{Hypersequent Classical Processes (\hcp)}
  \label{fig:hccp}
\end{figure*}

\begin{theorem}\label{thm:cp2hcp-typing}
  If $\seq[P]{\Gamma}$ in \cp, then $\seq[\mtf{P}]{\Gamma}$ in \hcp.
\end{theorem}
\begin{proof}
  By induction on the structure of the derivation of $\seq[P]{\Gamma}$.
    \begin{itemize}
  \item 
    Case \textsc{Cut}.
    We rewrite as follows:
    \begin{center}
      \scalebox{0.9}{\(
        \begin{array}{lcl}
          \AXC{$\seq[P]{\mathcal{G} \hsep \tmty{x}{A}, \Gamma}$}
          \AXC{$\seq[Q]{\mathcal{H} \hsep \tmty{x}{A^\bot}, \Delta}$}
          \NOM{Cut}
          \BIC{$\seq[\cpCut{x}{P}{Q}]{
          \mathcal{G} \hsep \mathcal{H} \hsep \Gamma, \Delta}$}
          \DisplayProof
          & \Rightarrow
          & \AXC{$\seq[\mtf{P}]{\mathcal{G} \hsep \tmty{x}{A}, \Gamma}$}
            \AXC{$\seq[\mtf{Q}]{\mathcal{H} \hsep \tmty{x}{A^\bot}, \Delta}$}
            \NOM{H-Mix} 
            \BIC{$\seq[\piPar{\mtf{P}}{\mtf{Q}}]{
            \mathcal{G} \hsep \mathcal{H} \hsep
            \tmty{x}{A}, \Gamma \hsep \tmty{x}{A^\bot}, \Delta}$}
            \NOM{H-Cycle}
            \UIC{$\seq[\piNew{x}{(\piPar{\mtf{P}}{\mtf{Q}})}]{
            \mathcal{G} \hsep \mathcal{H} \hsep \Gamma, \Delta}$}
            \DisplayProof
        \end{array}
        \)}
    \end{center}
  \item
    Case $(\tens)$.
    We rewrite as follows:
    \begin{center}
      \scalebox{0.9}{\(
        \begin{array}{lcl}
          \AXC{$\seq[P]{\mathcal{G} \hsep \tmty{y}{A}, \Gamma}$}
          \AXC{$\seq[Q]{\mathcal{H} \hsep \tmty{x}{B}, \Delta}$}
          \SYM{\tens}
          \BIC{$\seq[\cpSend{x}{y}{P}{Q}]{
          \mathcal{G} \hsep \mathcal{H} \hsep \tmty{x}{A \tens B}, \Gamma, \Delta}$}
          \DisplayProof
          & \Rightarrow
          & \AXC{$\seq[\mtf{P}]{\mathcal{G} \hsep \tmty{y}{A}, \Gamma}$}
            \AXC{$\seq[\mtf{Q}]{\mathcal{H} \hsep \tmty{x}{B}, \Delta}$}
            \NOM{H-Mix}
            \BIC{$\seq[\piPar{\mtf{P}}{\mtf{Q}}]{
            \mathcal{G} \hsep \mathcal{H} \hsep
            \tmty{y}{A}, \Gamma \hsep \tmty{x}{B}, \Delta}$}
            \NOM{H-Cycle}
            \UIC{$\seq[\piNew{x}{(\piPar{\mtf{P}}{\mtf{Q}})}]{
            \mathcal{G} \hsep \mathcal{H} \hsep \Gamma, \Delta}$}
            \DisplayProof
        \end{array}
        \)}
    \end{center}
  \item
    Case $(\one)$.
    We rewrite as follows:
    \begin{center}
      \scalebox{0.9}{\(
        \begin{array}{lcl}
          \AXC{}\SYM{\one}
          \UIC{$\seq[\cpHalt{x}]{\tmty{x}{\one}}$}
          \DisplayProof
          & \Rightarrow
          & \AXC{}
            \NOM{H-Halt}
            \UIC{$\seq[\piHalt]{\emptyhypercontext}$}
            \SYM{\one}
            \UIC{$\seq[\piBoundSend{x}{}{\piHalt}]{\tmty{x}{\one}}$}
            \DisplayProof
        \end{array}      
        \)}
    \end{center}
  \end{itemize}
  The other cases are trivial.
\end{proof}

\begin{definition}\label{def:hcp-bigparr}
  We can interpret environments as types by collapsing them using a series
  of pars.
  \begin{gather*}
    \begin{array}{lcl}
      \ty{\bigparr(\emptycontext)}
      & = & \ty{\bot}
      \\
      \ty{\bigparr(\tmty{x_1}{A_1} , \dots , \tmty{x_n}{A_n})}
      & = & \ty{A_1 \parr \dots \parr A_n}
    \end{array}
  \end{gather*}
\end{definition}

\begin{lemma}\label{lem:hcp-bigparr}
  If $\seq{\Gamma}$ in \hcp, then $\seq{\bigparr\Gamma}$ in \hcp.
\end{lemma}
\begin{proof}
  By induction on the structure of $\Gamma$.
  \begin{itemize}
  \item
    If $\ty{\Gamma} = \ty{A}$, then the statement is trivially true.
  \item
    If $\ty{\Gamma} = \ty{A, B, \Gamma'}$,
    then we apply $(\parr)$ to get $\seq{A \parr B, \Gamma'}$,\\
    and apply the induction hypothesis.
  \end{itemize}
\end{proof}

\begin{definition}\label{def:hcp-bigtens}
  We can interpret hyper-environments as types by collapsing them using a series
  of tensors. In the case of the empty hyper-environment
  $\ty{\emptyhypercontext}$, we interpret this as the unit of tensor.
  \begin{gather*}
    \begin{array}{lcl}
      \ty{\bigtens(\emptyhypercontext)}
      & = & \ty{\one}
      \\
      \ty{\bigtens(\Gamma_1 \hsep \dots \hsep \Gamma_n)}
      & = & \ty{\bigparr\Gamma_1 \tens \dots \tens \bigparr\Gamma_n}
    \end{array}
  \end{gather*}
\end{definition}

\begin{theorem}\label{thm:hcp2cp-bigtens}
  If $\seq{\mathcal{G}}$ in \hcp, then $\seq{\bigtens\mathcal{G}}$ in \cp.
\end{theorem}


\bibliographystyle{eptcs}
\bibliography{main}
\end{document}
%%% Local Variables:
%%% TeX-master: "main"
%%% End:
